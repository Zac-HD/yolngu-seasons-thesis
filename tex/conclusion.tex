\chapter{Contributions and Concluding Comments}
\label{ch:conclusion}

This thesis contributes to academic knowledge by the publication of novel
findings, development of methods to quantify Yolngu seasons, and demonstration
of a research approach which integrates Indigenous and scientific perspectives.
The research makes four distinctive contributions to scientific understanding
of Indigenous and Yolngu seasonal knowledge.

\begin{description}
\item[Calendar structure]
    Research participants describe seasons on three timescales with distinct
    monsoonal, meteorological, and ecological indicators for each.
    Yolngu seasons can occur in any order, even `interrupting' one another
    depending on indicator conditions such as wind direction.  This structure
    and detail is not explicitly evident in previous published research, and
    may generate new insights into the relationship between climate and
    ecology in Arnhem Land.

\item[Characterisation and definition of seasons]
    Yolngu seasons are defined by environmental observations rather than the
    passage of time or astronomical indicatiors.  Definitions may be in terms
    of weather, plant or animal dynamics, or a combination thereof.
    This finding builds on and extends the best previously published research.
    Participants agreed that weather indicators would be sufficient to
    determine which meteorological season occured on any given day.  The definitions of
    Yolngu seasons vary between locations -- as do the names and even number
    of seasons.

\item[Quantified seasons]
    An authentic Yolngu seasonal calendar was constructed based on literature
    and interview data.  These seasons were defined in terms of daily weather
    observations, and substantial analysis was undertaken -- including
    production of the first known timeseries identifying which season occured
    on each day, covering more than a decade of data at each of five weather
    stations.

\item[Characterising timing]
    Based on the ground-breaking quantification of seasons, this study is the
    first to precisely describe the timing of Australian Indigenous seasons
    (in this study for Yolngu seasons at Galiwinku and Milingimbi)
    rather than simply reporting which months
    are generally associated with a season.  This enables meaningful
    discussion of how to characterise the timing of such flexible seasons,
    and sets the stage for future work investigating fine relationships between
    Indigenous seaons and other climate or ecological data.
\end{description}

\clearpage

Uncertainty remains in the characterisation of seasonal instability or `jiter',
and the extent to which rapid changes in seasons (i.e. interruptions) in the
constructed record reflect Yolngu understanding is not clear.
In order to test whether the quantitative analysis of this study accurately
represents Indigenous knowledge, further data gathering through fieldwork or
remote surveys over one or more years would allow verification of the record
going forward.

The specific methods developed to quantify Yolngu seasons appear robust, but
difficult to generalise.  While it is in principle possible to so quantify
any calendar defined in terms of weather conditions, the qualitative
investigation into the structure and definitions of the calendar is unavoidable
for well-grounded studies in this area.  These methods provide an
avenue for further research, \emph{as part of a larger programme} -- not a
silver bullet.

This thesis demonstrates a flexible methodology for integration
of Indigenous and quantitative scientific knowledge and knowledge systems,
which could be applied in domains beyond seasonal or ecological knowledge.
It finds the general approach -- participant-led conversations followed quantitative
application of insights -- surfaces novel and unexpected information, and
makes it useful across cultural boundaries.

Further research is recommended on Indigenous seasonal calendars, or
integrating Indigenous and non-Indigenous knowledge about other topics.  Such
research would be highly novel and applicable to many of the challenges facing
Australia in the twenty-first century, helping Indigenous and non-Indigenous
people to understand and adapt to global change, and perhaps for non-Indigenous
Australians to reach a genuinely Australian understanding of what it means to
live on this continent, our home.

