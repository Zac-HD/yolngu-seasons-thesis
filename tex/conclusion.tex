\chapter{Conclusion}
\label{ch:conclusion}


\section{Questions, Aims, Findings}

This thesis contributes to academic knowledge by the publication of novel
findings, development of methods to quantify Yolngu seasons, and demonstration
of a research approach which integrates Indigenous and scientific perspectives.

The research questions proposed in \cref{ch:introduction} were addressed, and
resulted in the four distinctive contributions to scientific understanding
of Indigenous and Yolngu seasonal knowledge below:

\begin{description}
\item[Calendar structure]
    Research participants describe seasons on three timescales with distinct
    monsoonal, meteorological, and ecological indicators for each.
    Yolngu seasons can occur in any order, even `interrupting' one another
    depending on indicator conditions such as wind direction.  This structure
    and detail is not explicitly evident in previous published research, and
    may generate new insights into the relationship between climate and
    ecology in Arnhem Land.

\item[Characterisation and definition of seasons]
    Yolngu seasons are defined by environmental observations rather than the
    passage of time or astronomical indicatiors.  Definitions may be in terms
    of weather, plant or animal dynamics, or a combination thereof.
    This finding builds on and extends the best previously published research.
    Yolngu participants agreed that weather indicators would be sufficient to
    determine which meteorological season occured on any given day.  The definitions of
    Yolngu seasons vary between locations -- as do the names and even number
    of seasons.

\item[Quantified seasons]
    A Yolngu seasonal calendar was constructed based on literature
    and interview data.  Seasons were defined in terms of daily weather
    observations, and substantial analysis was undertaken -- including
    production of the first known timeseries identifying which season occured
    on each day, covering more than a decade of data at each of five weather
    stations.

\item[Characterising timing]
    Based on the ground-breaking quantification of seasons, this study is the
    first to precisely describe the timing of Australian Indigenous seasons
    (in this study for Yolngu seasons at Galiwinku and Milingimbi)
    rather than simply reporting which months
    are generally associated with a season.  This enables meaningful
    discussion of how to characterise the timing of such flexible seasons,
    and sets the stage for future work investigating fine relationships between
    Indigenous seaons and other climate or ecological data.
\end{description}

Uncertainty remains in the characterisation of seasonal instability or `jiter',
and the extent to which rapid changes in seasons (i.e. interruptions) in the
constructed record reflect Yolngu understanding is not clear.
In order to test whether the quantitative analysis of this study accurately
represents Indigenous knowledge, further data gathering through fieldwork or
remote surveys over one or more years would allow verification of the record
going forward.


The pursuit of these findings was motivated by the wider aim of developing
an approach to integrating traditional seasonal knowledge with a quantitative
analytical approach.  The research acheived this aim.

The specific methods developed to quantify Yolngu seasons appear robust, but
difficult to generalise.  While it is in principle possible to so quantify
any calendar defined in terms of weather conditions, the qualitative
investigation into the structure and definitions of the calendar is unavoidable
for well-grounded studies in this area.  These methods provide an
avenue for further research, especially as part of a larger programme
investigating Indigenous seasonal calendars.

This thesis demonstrates a flexible methodology for integration
of Indigenous and quantitative scientific knowledge and knowledge systems,
which could be applied in domains beyond seasonal or ecological knowledge.
It finds the general approach -- participant-led conversations followed quantitative
application of insights -- surfaces novel and unexpected information, and
makes it useful across cultural boundaries.



\section{Directions for Further Study}
\label{sec:further-study}

Having demonstrated the viability of quantitative analysis of Indigenous
seasons, and discussed the value of such research, there are three distinct
directions for further study:  expanding one or more of the depth, breadth,
or topic of the research.  Alternative research questions could also address the aim
of this research -- \textit{``developing an approach to integrating traditional
knowledge with a quantitative analytical approach''} -- but are not suggested
here.

\subsection{Depth of Study}
Deeper studies would likely focus on Indigenous Knowledge as it
relates to seasonal patterns, and work to integrate IK with Western-style scientific
knowledge.  Researchers must ensure that both Indigenous and non-Indigenous
people benefit from such work, and not continue the history of exploitative
`sharing' condemed by eg. \citet{smith1999}.  Further analysis of monsoon-
or weather-based seasons would also fit in this category.

\begin{description}
\item[Attention to Yolngu benefits]
    Drawing on two-ways research, future studies should ensure that Indigenous
    people are involved in setting goals and implementing outcomes that are
    important to them, as well as benefiting non-Indigenous researchers.

\item[Kinds of seasons]
    \Cref{sec:complex-seasons} describes a typology of seasons -- monsoonal,
    meteorological, and ecological -- that can be identified as part of a
    `single' Yolngu calendar (to the degree that `single Yolngu calendar'
    is a coherent idea).  Further investigation of the conceptual basis of this typology would
    be enlightening, as would gaining a stronger understanding of how
    seasonal indicators interact.

\item[Ecological knowledge]
    Investigating Indigenous ecological seasons could provide deep insights
    into local ecology and traditional natural resource management. The
    principal challenges are logistical:  such study would require substantial
    investment in relationship-building, and collecting the data
    required to `translate' Indigenous Knowledge to Western science would
    be expensive.

\item[Per-year patterns]
    There is clearly a degree of variation in the ways in which Indigenous seasons occur between
    different years; studying these variations may reveal patterns.  Are there distinct
    regimes that describe seasonal onset, timing, or interruption?  If so, what
    data are required to categorise a year by regime and to make useful predictions?
\end{description}

\subsection{Breadth of Study}
Expanding the breadth of study would result in similar but less exploratory
research; essentially trading novelty for rigor.  The clearest avenue for
this is to collect and analyse more data from an increased number of sites.

\begin{description}
\item[Multiple Indigenous calendars]
    Replication of this study with in other areas to document other Indigenous calendars would be
    valuable in each case, and the ability to compare calendars and investigate
    possible correlations between local climate and how seasons are defined
    could yield fascinating insights.

\item[Single calendar, multiple sites]
    Similarly, comparing the calendar of a single language group across
    multiple sites and subtly different climates could be very interesting.
    An `objective' weather record could really help researchers to draw out
    the differences between calendars, and perhaps see some of the subtleties.
    While this study had participants and weather observations from multiple
    sites, it did not analyse them as qualitatively distinct records or
    attempt any comparisons.

\item[Iterative fieldwork]
    Future work should if at all possible invest the time and resources
    neccessary for multiple visits to the study area.  Second opinions,
    feedback on quantitative results, and strong relationships supported by
    long-term engagement can all be valuable.
\end{description}

\subsection{Topic of Study}
Expanding the topic of the study might involve considering links to the wider
context, for example climate indices and long-term change, or non-Indigenous
calendars.

\begin{description}
\item[Non-Indigenous calendars]
    A comparative study of many kinds of calendars could be fascinating --
    Indigenous, agricultural, solar, lunar, or harvest seasons -- unlocking
    the knowledge embedded in each.

\item[Climatology]
    With seasons defined by weather observations, a climatological study of the
    Yolngu calendar could suggest links between forecastable climate indices
    and Arnhem Land ecology.  Research could begin by investigating correlations
    between seasonal occurance and non-surface weather, or look at cycles with
    timescales on the order of months (eg. the Madden-Julian Oscillation), years (ENSO,
    the Indian Ocean Dipole), or even decades (the Pacific Decadal Oscillation).

\item[Climate Change]
    The methods used in this study can be applied to any dataset which
    includes a daily record of the relevant variables.  This enables
    investigation of observed, estimated, or projected past and future changes
    across scenarios including a wider range of conditions than exist in the historical record.
    One obvious application of this is to investigate how Yolngu seasons have
    changed over the past century, and how they may change under future climate
    change scenarios in model forecast and backcasts.  Yolngu participants
    indicated substantial interest in such studies.
\end{description}


\clearpage
\section{Implications}

Beyond specific applications, this research has benefits which are less tangible.
It demonstrates more nuanced ways of thinking about tropical seasons than the
simplistic Wet/Dry dichotomy \citep{willmett2009}, and builds the foundation
for new framing around climate change -- everyone has lived experience with
weather and seasons, while changing global mean temperature is an abstract and
distant concept.
%
In the long term, it may contribute to the long and difficult process of
non-Indigenous Australians learning to understand, live in, and nurture
Australia through a fully Australian world-view.  This would require setting
aside our `maladaptive traditions' grounded in non-Australian contexts
\citep{flannery1994} to celebrate the unique challenges presented by our variable and extreme
environments.  Imagine a future where local radio discusses not the timing of
European flowers, but seasons -- with a guest panel of Indigenous elders and
climate scientists.
%
If this research can inspire the building of bridges or the breaking down of walls
between our many seperate forms of knowledge, Australia may be better placed
to meet the challenges of the coming century than the last.

