\chapter{Conclusions}
\label{ch:conclusion}

This thesis contains a number of novel findings, developed an approach
to seasonal knowledge which integrates Indigenous and scientific perspectives,
and makes several contributions to academic knowledge.


\section{Key Findings}

\begin{description}
\item[Calendar structure]
    Research participants describe seasons on three timescales with distinct
    monsoonal, meteorological, and ecological indicators for each.  This
    structure is not evident in previous research, and may generate new
    insights into the relationship between climate and ecology in Arnhem Land.
    Yolngu seasons can occur in any order, even `interrupting' one another
    depending on indicator conditions such as wind direction.

\item[Season definitions]
    Yolngu seasons are defined by environmental observations rather than the
    passage of time or astronomical indicatiors.  Definitions may be in terms
    of weather, plant or animal dynamics, or a combination thereof.
    Participants agreed that weather indicators would be sufficient to
    determine which season occured on any given day.  The definitions of
    Yolngu seasons vary between locations -- as do the names and even number
    of seasons.

\item[Quantified seasons]
    An authentic Yolngu seasonal calendar was constructed based on literature
    and interview data.  These seasons were defined in terms of daily weather
    observations, and substantial analysis was undertaken -- including
    production of the first known timeseries identifying which season occured
    on each day, covering more than a decade of data at each of five weather
    stations.

\item[Characterising timing]
    Based on the groundbreaking quantification of seasons, this study is the
    first to describe the timing of Yolngu -- possibly Australian Indigenous --
    seasons with greater precision than a qualitative claim as to which months
    are generally associated with a season.  This result enable meaningful
    discussion of how to characterise the timing of such flexible seasons,
    and sets the stage for future work investigating fine relationships between
    Indigenous seaons and other climate or ecological data.

\end{description}


\section{Developed Methods}
The quantitative methods constructed are applicable to...
They are [not] robust and generalisable...



\section{Contribution}
This research makes a novel contribution in the following areas...



