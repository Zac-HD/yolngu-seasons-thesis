\chapter{Conclusions}
\label{ch:conclusion}

This thesis contributes to academic knowledge by the publication of novel
findings, development of methods to quantify Yolngu seasons, and demonstration
of a research approach which integrates Indigenous and scientific perspectives.
%
Four areas of key findings in relation to Yolngu seasonal knowledge are:

\begin{description}
\item[Calendar structure]
    Research participants describe seasons on three timescales with distinct
    monsoonal, meteorological, and ecological indicators for each.  This
    structure is not evident in previous research, and may generate new
    insights into the relationship between climate and ecology in Arnhem Land.
    Yolngu seasons can occur in any order, even `interrupting' one another
    depending on indicator conditions such as wind direction.

\item[Season definitions]
    Yolngu seasons are defined by environmental observations rather than the
    passage of time or astronomical indicatiors.  Definitions may be in terms
    of weather, plant or animal dynamics, or a combination thereof.
    Participants agreed that weather indicators would be sufficient to
    determine which season occured on any given day.  The definitions of
    Yolngu seasons vary between locations -- as do the names and even number
    of seasons.

\item[Quantified seasons]
    An authentic Yolngu seasonal calendar was constructed based on literature
    and interview data.  These seasons were defined in terms of daily weather
    observations, and substantial analysis was undertaken -- including
    production of the first known timeseries identifying which season occured
    on each day, covering more than a decade of data at each of five weather
    stations.

\item[Characterising timing]
    Based on the groundbreaking quantification of seasons, this study is the
    first to describe the timing of Yolngu -- possibly Australian Indigenous --
    seasons with greater precision than a qualitative claim as to which months
    are generally associated with a season.  This result enable meaningful
    discussion of how to characterise the timing of such flexible seasons,
    and sets the stage for future work investigating fine relationships between
    Indigenous seaons and other climate or ecological data.
\end{description}

\clearpage

The specific methods developed to quantify Yolngu seasons are robust, but
difficult to generalisable.  While it is in principle possible to so quantify
any calendar defined in terms of weather conditions, the qualitative
investigation into the structure and definitions of the calendar is unavoidable
for well-grounded studies in this area.  These methods provide a fascinating
avenue for further research, \emph{as part of a larger programme} -- not a
silver bullet.

This thesis demonstrates a flexible methodology for integration
of Indigenous and quantitative scientific knowledge and knowledge systems,
which could be applied in domains beyond seasonal or ecological knowledge.
The general approach -- participant-led conversations followed quantitative
application of insights -- surfaces genuinely surprising information, and
makes it useful across cultural boundaries.

Further research on Indigenous seasonal calendars, or integrating Indigenous
and non-Indigenous knowledge about other topics, is strongly encouraged.  Such
research would be highly novel and applicable to many of the challenges facing
Australia in the twenty-first century -- not least for migrant Australians to
reach a genuinely Australian understanding of what it means to live on this
continent, our new home.

