\chapter{Context and importance of this research}
\label{ch:context}
This chapter covers the basic context of the research,
regarding the location and past studies of the Yolngu calendar;
finds a gap in the literature and surveys relevant fields;
and lays out the overarching methodology of and strategy for this research.


\section{Context}
This section is at notes/draft stage.

\begin{figure}[h]
    \centering
    \includegraphics[width=\textwidth]{mapdraft.png}
    \caption[Map showing the study area, NE Arnhem Land]{
        Map showing the study area, NE Arnhem Land.
        Analysis in \autoref{ch:quantify} uses weather observations from several of the towns shown.
        }
    \label{fig:arnhem-map}
\end{figure}

Figure~\ref{fig:arnhem-map} shows the study area in north-east Arnhem Land in Australia.

The coastal north-east of Arnhem Land belongs to the Yolngu people,
with one of the oldest living cultures on Earth.

In a letter inviting collaboration on this research (see \autoref{sec:ethics}),
a senior Yonlgu man explained that:

\blockquote{
    ... Yolngu have made careful observation of the ways of nature and the
    seasons over the millenia and have passed on that knowledge down the
    generations.  We know the changes that are taking place in the seasons...
}

Yolngu also have a tradition of engagement across cultures, including
the Macassan trade with Indonesia in the 1700s and engagement with Methodist
missionaries through the 1900s.

The Yirrkala Bark Petition (\autoref{fig:bark-petition}) marks a crucial
point in the Australian land rights movement, recognised by indigenous and
non-indigenous people alike.  It is also prone to misinterpretation by those
who see a document in two languages with a decorative border: the border
is the most important part!  (see caption, expand this section)



\begin{wrapfigure}{R}{0.5\textwidth}
    \includegraphics[width=\linewidth]{bark-petition.jpg}
    \caption[The Yirrkala Bark Petition]{
        The Yirrkala Bark Petition, 1963.
        The Bark Petition concerns the bauxite mining lease granted at Nhulunbuy
        without consultation with Yolngu, and helped kick-start the Australian
        land rights movement.

        The petition is presented in three ways: text in English,
        text in Yolngu Matha, and in the border.
        This painting is not a decoration; it is a reproduction of
        the title deeds to the ancestoral estates of the signatories -
        and the most important part of the document.
        }
    \label{fig:bark-petition}
\end{wrapfigure}



\section{Literature Review}
\label{sec:lit-review}

After -detailed search strategy-, I was unable to identify any studies which quantified indigenous Australian seasons.
I thus review a number of related fields, explaining the relevance of each and drawing together a summary of this 'gap at the intersection'.
MUST have table here pointing to other parts of the literature review, explaining that concrete parts are in the related methods section to avoid misunderstanding. 
Possible sections – to be reorganised and mutated:

\begin{itemize}
\item Cross cultural and qualitative methods
\item Tropical seasonality, including monsoon and ENSO
\item Anticipated benefits and applications of the research
\item Relationship to phenology and historical climate change
\item Indigenous seasons literature (summary, also compare and contrast)
\item Overview of calendars and seasons – what are they and how are they defined?
\end{itemize}

\section{Methodology and Research Framework}

My guiding research questions are:
\begin{itemize}
\item What Yolngu seasonal calendars are generally recognised?
\item Which best affords cross-cultural collaboration?
\item What are the properties and changes that define this calendar?
\item How may these seasons be characterised with respect to temporal and physical patterns in climate?
\item What, if anything, about these seasons is changing in the historical record?
\item What might happen under future climate change?
\end{itemize}

I build on existing literature and primary research by deriving a physical characterisation of the Yolngu calendar,
combining traditional Indigenous knowledge with numerical climate science.
Using threshold values in meteorological variables rather than a predefined date range enables
analysis of means and variation in the timing of onset, duration of seasons, and annual patterns in climatological variables.
Preliminary work suggests that these definitions can also be applied to reconstructed or modelled climate data,
allowing investigation of trend or abrupt changes over time.
Well-founded definitions may allow scenario-based investigation of a wider range of conditions than exist in the historical record.



