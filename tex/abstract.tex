\chapter{Abstract}

Australia presents a challenging environment with highly variable climates.
Better ways of understanding and managing our natural environment are urgently needed, and
may be offered in the rich knowledge of Indigenous Australians -- knowledge that
is too often dissmissed, ignored, and undervalued.
%
This thesis develops an approach which integrates traditional seasonal knowledge
with quantitative science, demonstrated by investigating Yolngu
seasons in north-east Arnhem Land.

Where previous studies of Indigenous seasons and calendars in Australia have
recorded weather, social, and ecological characteristics of seasons -- treating
them as objects of study -- this thesis treats Yolngu seasons as a framework
for study, working with Yolngu participants to create a quantified calendar
based on the observational weather record.  No known previous research has quantified
Australian Indigenous seasons in this way.


The methodology is cross-disciplinary and exploratory, drawing on the two-ways
research literature.  Informal, participant-led interviews with Yolngu people
provide qualitative descriptions of the seasons and the structure of the
calendar, and advice on how the seasons might be analysed.
%
The key findings of this thesis are qualitative, some directly from interviews
-- eg. the structure of Yolngu calendars and definitions of the seasons; while
others are the results of numerical analysis -- quantifying and characterising the seasons.


Yolngu participants describe monsoonal, meteorological, and ecological
seasons with distinct time-scales and indicators.  Seasons can occur in any
order, even `interrupting' one another depending on indicator conditions.
These properties are not recorded by previous research.
%
Yolngu seasons are defined by environmental observations rather than the
passage of time, the specifics of which vary substantially with location --
as do their names and even what seasons are recognised.  Detailed qualitative
descriptions were collected to create a seasonal calendar.
%
Each season was described in terms of daily weather and these definitions
applied to the observational data, deriving the first known timeseries of
season occurance.  This dataset allows characterisation of the seasons,
and the first ever quantitative investigation of their timing.  Accurate
characterisation of this timing is another contribution of this research.

The methodology and methods developed in this thesis are suitable for use in
similar research elsewhere.  The code which quantifies Yolngu seasons in terms of weather
conditions is robust and reusable, and could be adapted for study of the
qualitative structure and definitions of other calendars.
Flexible methodology for integration of Indigenous and quantitative
scientific knowledge and knowledge systems is applicable in domains far beyond
seasonal or ecological knowledge.  Participant-led conversations followed by
quantitative investigation uncovers and highlights unanticipated information
and perspectives across cultural boundaries.


This thesis contributes to academic knowledge by developing methods to
quantify Yolngu seasons, and demonstrating a research approach which
integrates Indigenous and scientific perspectives.
%
The outcomes suggest fruitful avenues for future research, which may consider seasonal calendars in
other location or in greater depth or integrate Indigenous and non-Indigenous
knowledge about other topics.  These studies could extend scientific knowledge
and understanding, as well as supporting concrete applications.  In either case
they would contribute to a novel and genuinely local understanding of
Australia, with implications for land and natural resource management,
decision making, and policy.

