\chapter{Abstract}

Indigenous seasonal calendars encounter valuable traditional knowledge, through their very definition:
where Western seasons are determined by solar dates and invariant across continents,
indigenous seasons are defined by local conditions.

Indigenous seasons vary from place to place, are recognised by weather or ecological events,
take place over varying timescales, and occur at different times of year depending on environemtal factors.

This thesis characterises a Yolngu seasonal calendar, from Galiwinku in north-east Arnhem Land, Australia.

The calendar is derived from a series of interviews with Yolngu people and elders, and
non-indigenous teachers, mission workers, and scientists working in relevant field.
The particular seasonal cycle is selected for affordance of cross-cultural study and understanding,
and I discuss my experience of ambiguity in the meaning of `season' and `calendar'.

I then compare two methods for quantitative recognition of seasons based on weather records
and qualitative descriptions from research participants.  The first is objectively replicable,
deriving threshold values from the historical record for months typically associated with each season.
The second is more precise, identifying onset in periods of rapid trend change in variables.
This exploratory analysis results in a `quantified calendar' mapping seasons to meteorology,
a result unprecedented in the surveyed literature.

Enabled by the quantified calendar I derive further novel results, including:
\begin{itemize}
\item typical values and variation in onset date, duration, and weather conditions for each season,
\item correlation between Yolgu seasons and phenomena such as ENSO and the monsoon,
\item decadal analysis of changes to Yolngu seasons over the historical record,
\item forecast scenarios for Yongu seasons under climate change (based on existing models), including decadal analysis,
\item and, (as practical and if possible) discussion of what this might mean for Yolngu people and lifestyles.
\end{itemize}




