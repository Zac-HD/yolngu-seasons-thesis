\chapter{Methods}
\label{ch:methods}

\section{Methodology}

My guiding research questions are:
\begin{itemize}
\item How are Yolngu seasonal calendars defined, and does this vary?
\item Which Yolngu calendar is best suited for mixed-methods study?
\item What are the properties and changes that define this calendar?
\item How may these seasons be characterised by meterological parameters?
\item Is anything about these seasons changing in the historical record?
\end{itemize}

I build on existing literature and primary research by deriving a physical
characterisation of the Yolngu calendar, combining traditional Indigenous
knowledge with numerical climate science. Using threshold values in meteorological
variables rather than a predefined date range enables analysis of means and
variation in the timing of onset, duration of seasons, and annual patterns
in climatological variables.

Preliminary work suggests that these definitions can also be applied to
reconstructed or modelled climate data, allowing investigation of trend
or abrupt changes over time. Well-founded definitions may allow scenario-based
investigation of a wider range of conditions than exist in the historical record. 

~\\

TODO - fill out methodology

Historical climatology is important in that it informs my methods,
but also for the concept and to contextualise the research.  May lead to 
some comparisons between `lay knowledge' and IEK (an important point).\\



\subsection{Fieldwork and interviews}

\begin{itemize}
\item Why I'm doing primary data collection, why fieldwork
\item Why personal conversations with minimal structure, why avoiding surveys etc.
\item Explain cultural sensitivities etc
\end{itemize}


Existing research on Australian Indigenous Calendars has generally been 
published on posters or occasionally in books, and focusses on passively 
recording and reporting Indigenous knowledge of seasonality on its own terms.  

My primary motivation for fieldwork and primary data collection is thus that 
the existing data is not well suited to the inquiries I wish to make – not just 
how the calendar is defined, but also to elicit the knowledge required to 
independently recognise the seasons.  This includes historical questions about 
past seasons and climate, hypotheticals about what the interpretation of 
unlikely events might be, and many others listed below.

Rather than structured interviews or a written survey, I conduct minimally-
directed conversations with research participants.  Bringing avoidable 
paperwork to an indigenous community is generally not culturally appropriate, 
as it can devolve into mutual frustration rather than mutual learning.  An 
assumption of literacy may exclude potential participants where English is 
often a fourth (or sixth, or later!) language.

Calendars and seasons are a topic which is not usually secret business, though 
there may be particular sacred topics associated with them.  Especially with an 
explicit invitation to come and learn, there is little risk of inappropriate 
sharing of information.

Yolngu have long memories and significant knowledge of their 
calendar.  People in the community can share their experience and different 
perspectives.  Elders bring an important perspective, but so do younger people 
and those engaged in Ranger groups or related work.



\subsection{Bridging Qualitative and Quantitative methods}
This is a vital area, particularly when working across cultures too.
TODO - discuss approach, how exploratory qual becomes the foundation, etc.



\subsection{Numerical Analysis}
There's nothing particularly fancy in here, 






\section{Research with Indigenous People}


\subsection{Relationship-building}
Discuss the relationship-building stage in some detail; it's really important.
Relationships are a key part of cross-cultural research.  See Sue Smith, as 
Sean has mentioned.

This collaborative and cross-cultural approach has been and will be a 
significant influence on the direction of the research. (explore it more then)

%% This is a particularly important point
Without a high level of engagement, including introductions and logistical
assistance with travel permits, it would be impractical or impossible to conduct
such research for an Honours thesis.


\subsection{Two-ways research}
Broaden approach; not just two-ways – read other research methodology and 
situate work.  Really really need to do this.

Around the world, and more recently in Australia, there has been growing 
interest in `two-ways' research \citep{turner2009,prober2011}, 
which \blockquote{uses combinations of Indigenous and non-Indigenous knowledge and 
methods, and with the involvement of both Indigenous and non-Indigenous people 
towards a common goal} \citep{ens2014}.  

However, working across knowledge systems can be difficult for both Indigenous 
and non-Indigenous people, and often challenges institutions and research 
funding bodies to invest substantial time and resources in relationship-
building with uncertain outcomes.  The Ngan'gi Seasons Calendar emphasises 
ecological events and customary activities for each season above the timing and 
meteorological conditions.  Produced in collaboration between Ngan'gi knowledge 
holders and the CSIRO, `success' required a long-term commitment to work in 
remote areas on the Daly River, a high degree of risk tolerance in 
research funding, and many years work building solid relationships \citep{woodward2010}.

TODO - A bit thin; expand benefits and value of knowledge

When \citet{petheram2010} investigated perspectives on climate change 
adaptation in NE Arnhem Land; the community engaged in terms of resilience to 
existing issues rather than adaptation to climate change, challenging the basis 
of their research.  With severe problems related to poor housing, drug abuse, 
and extensive mining land degradation, climate change is of little concern to 
participants who mostly attribute `strange changes' to local environmental 
damage \citep{green2010a}.




\section{Qualitative and Interview Methods}

I rely on four sources of qualitative knowledge and context for the Yolngu
seasons, in order of importance:

\begin{itemize}
\item Interview with Yolngu people form the basis of my qualitative research, and
        are the definitive source of information about Yolngu seasons.
\item Interviews and discussion with non-Indigenous researchers or teachers experienced
        in remote communities help contextualise this knowledge, and warned of
        common misinterpretations - as well as pointing out nuance.
\item Published literature, particularly around cross-cultural research \citep[eg.][]{smith1999},
        Australian Indigenous seasons \citep[eg.][]{prober2011,oconnor2010}, and Yolngu
        seasons directly \citep{davis1989}.
\item Grey literature, such as posters produced for use in remote schools, workbooks
        for cross-cultural teacher training, etc.
\end{itemize}


I conducted a series of informal, semistructured interviews discussing
seasons and climate. Recruitment was primarily by a `snowball' pattern,
where participants were asked to suggest further potential participants.

The main `seeds' of this pattern were personal contacts, some Yolngu and some
non-Indigenous people who have spent decades living and working in remote
communities. I followed several disconnected threads such as the
\citet{CSIROcals} project, which helped to clarify and test my interpretation
of the interviews.

See \autoref{sec:ethics} for details of the human ethics approval.
Semi-structured interviews with Yolngu participants are my primary source
of information.  These interviews are utterly indispensible.
The literature contextualises and informs the interviews, and suggested many fruitful
directions for direct questions or later searches for documents.

The most consistently productive interviews were focussed on teaching about
seasons: names and definitions, typical conditions and timing, and touching on
extreme events and outliers. Thematic questions include:
\begin{itemize}
\item What are the names of the seasons?
\item When does [a season] usually occur?  How do you know when it starts (definition)?
\item How long does [a season] last?  What weather or events usually occur in this season?
\item Do you think the seasons have changed over your lifetime?  Why/why not?  How can you tell?
\item Are there some years where a season is skipped?  What happens?
\item Do you remember any unusual events?  What happened?
\item What might the calendar be like if [example climate impact] happened? 
      (eg changes to wind, temperature, rainfall patterns) 
\item What are some important ways you use the land which depend on seasons?
        Have these changed over time?  Why?
\end{itemize}

Following the interviews, I also engaged in substantial reflection on the nature
of my questions and the ambiguity of the responses and data I collected.
In this process, allowing participants to discuss whatever they felt relevant
is a key way to remain open to unexpected information - including insights
which changed my understanding of what a Yolngu seasonal calendar was!






\section{Bridging Approaches to Quantify a Calendar}


\paragraph{Observing seasonal onset}~\\
There is a considerable body of literature dealing with seasons defined
by observed environmental change - from temperature thresholds in
Scandinavia to 'leaf out' for botanists or butterflies for zoologists.

While such definitions are highly dependent on context, some common
elements can be observed.  

Monotonic changes and qualitative changes are easiest to pin down.
Ie binary states, and/or only passing the threshold once per year.

TODO more on this, add references.

I've started reading the literature on historical climatology, but don't have 
enough depth to write much yet.  It's going to be an important angle though. 

Matching oral history with numerical data is an established technique in 
climate science, as recorded observations can greatly assist in reconstruction 
of conditions before the instrumental record began.  In this case, observations 
within living memory are an excellent source, as the instrumental record is 
even today very sparse in Arnhem Land.  Investigation of social adaptation to 
local climate is an interesting and often closely linked field.

Look more into non-observational records; as far as I know dendrology etc has 
not been done much in the region; this is relevant as a missing alternative to 
oral histories.


\paragraph{Yolngu participants} made several comments which are not about
the seasonal calendar itself, but are of vital importance when quantifying
it - teaching about how exactly to recognise each season.

Seasons are not limited to a particular order, and may happen one or more
times per year (by definition zero or more; historically at least once).
Any season may be `interrupted' by another if the latter's characteristic
conditions hold for longer than two or three days.
Where non-indigenous people might discuss a `cold snap', Yolngu might
say that winter had interrupted summer for a few days!

Wind plays a defining role in Yolngu seasons, but not all winds are
seasonal indicators.  In coastal communities, the day-time and especially
afternoon wind is dominated by the sea breeze, and carries relatively little
seasonal information.  Participants recommended considering wind at 9am
and 6pm, saying that at 3pm (the meteorological standard) the sea breeze
overwhelms the seasonal changes.

TODO check notes for other issues.


\paragraph{Weather observations} were generously supplied by the
Australian Bureau of Meterology.

\begin{table}[h]
    \centering
    \begin{tabular}{lllllll}
        Station no.  &  Name                 &  Location     &  Latitude   &  Longitude   &  Elevation  &  Opened   \\
%                                                                                                      
        014401       &  WARRUWI AIRPORT      &  Warruwi      &  11.6500S   &  133.3797E   &  19.2       &  Jan 1916 \\
        014404       &  MILINGIMBI AIRPORT   &  Milingimbi   &  12.0932S   &  134.8919E   &  15.0       &  Mar 2003 \\
        014405       &  MANINGRIDA AIRPORT   &  Maningrida   &  12.0569S   &  134.2339E   &  28.1       &  Oct 2003 \\
        014508       &  GOVE AIRPORT         &  Nhulunbuy    &  12.2741S   &  136.8203E   &  51.6       &  Jan 1944 \\
        014517       &  NGAYAWILI            &  Galiwinku    &  11.9971S   &  135.5726E   &  8.1        &  Oct 1999
    \end{tabular}
    \caption{Summary description of weather stations used.
        Locations are visible in \autoref{fig:arnhem-map}}
    \label{tab:weather-station-summary}
\end{table}

\autoref{tab:weather-station-summary} shows the name, ID, and location of
the weather stations used.

The weather variables of interest are:
\begin{itemize}
\item Rainfall in the 24 hours before 9am (local time), in milimeters.
\item Maximum temperature in the 24 hours after 9am (local time), in Degrees C.
\item Minimum temperature in the 24 hours before 9am (local time), in Degrees C.
\item Humidity measured as average daily dew point temperature, in Degrees C.
\item Wind speed measured in kilometers per hour, at 9am and 6pm local time.
\item Wind direction recorded as 16 compass points, at 9am and 6pm local time.
\end{itemize}

The data are cleaned by discarding observations accumulated over multiple days.
Observations which have been quality-controlled by the BoM and are considered
`wrong', `suspect', or `inconsistent with other known information' are discarded.
Observations which have not been assessed are retained.



\section{Quantitative Analysis}

Data analysis and presentation is entirely via my own Python scripts,
using a variety of scientific and data-analysis libraries.
I lean heavily on Pandas, Seaborn, NumPy, and Matplotlib.
For details and to read the -- well commented -- code, see
\autoref{sec:appendix-code}


I use two statistical approaches to detect seasonal onset.

The first is to calculate threshold levels, chosen based on typical conditions
in months which were mentioned as typical for each season in my interviews.
TODO - add detail on how to determine typical - likley to be a percentile or something.

The second is to calculate the local trend in each variable, identifying seasonal
change when on the normalised rate of change is high and the sign is correct.
TODO - more details on this too.

Each of these methods assigns a confidence rating for each season to each day of
observations.  These ratings are then reconciled to a single season, or unknown,
for each day.  The reconciliation draws on participants' comments to drop inconsistent
ratings before comparing normalised season ratings.


TODO - discuss handling of missing data and how (if) filled in.
Data is considered missing of it is not provided, has been through
quality assurance and found wanting, or was collected over seveal days.
Missing data is propogated throughout the analysis and not filled with other values.



\begin{itemize}
\item Calculation of summary stats for each season – typical and variation in conditions, onset, etc
\item Comparison of stations, other data as inputs to season recognition.
\item Normalisation functions from observations to gridded data inc. model backcasts if applicable.
\item Correlation to ENSO, monsoon, dipole moment, etc
\end{itemize}

Note that even the simplest of these is really novel, as nobody has
been able to calculate the summaries before!

The application element is conceptually simple:  based on the physical
descriptors and numerical data, quantitative characterisations of the
Yolngu seasons can be calculated.  These will include both typical values
and variation in season onset date, duration, temperatures, rainfall,
wind, and so on.





