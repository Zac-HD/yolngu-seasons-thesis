\chapter{Methods}
\label{ch:methods}

\section{Methodology}

I build on existing literature and primary research by deriving a physical
characterisation of the Yolngu calendar, combining traditional Indigenous
knowledge with numerical climate science. Using threshold values in meteorological
variables rather than a predefined date range enables analysis of means and
variation in the timing of onset, duration of seasons, and annual patterns
in climatological variables.

Preliminary work suggests that these definitions can also be applied to
reconstructed or modelled climate data, allowing investigation of trend
or abrupt changes over time. Well-founded definitions may allow scenario-based
investigation of a wider range of conditions than exist in the historical record. 

~\\

TODO - fill out methodology

Historical climatology is important in that it informs my methods,
but also for the concept and to contextualise the research.  May lead to 
some comparisons between `lay knowledge' and IEK (an important point).\\





\section{Research with Indigenous People}



Existing research on Australian Indigenous Calendars has generally been 
published on posters or occasionally in books, and focusses on passively 
recording and reporting Indigenous knowledge of seasonality on its own terms.  

My primary motivation for fieldwork and primary data collection is thus that 
the existing data is not well suited to the inquiries I wish to make – not just 
how the calendar is defined, but also to elicit the knowledge required to 
independently recognise the seasons.  This includes historical questions about 
past seasons and climate, hypotheticals about what the interpretation of 
unlikely events might be, and many others listed below.

Rather than structured interviews or a written survey, I conduct minimally-
directed conversations with research participants.  Bringing avoidable 
paperwork to an indigenous community is generally not culturally appropriate, 
as it can devolve into mutual frustration rather than mutual learning.  An 
assumption of literacy may exclude potential participants where English is 
often a fourth (or sixth, or later!) language.

Calendars and seasons are a topic which is not usually secret business, though 
there may be particular sacred topics associated with them.  Especially with an 
explicit invitation to come and learn, there is little risk of inappropriate 
sharing of information.

Yolngu have long memories and significant knowledge of their 
calendar.  People in the community can share their experience and different 
perspectives.  Elders bring an important perspective, but so do younger people 
and those engaged in Ranger groups or related work.


\subsection{Relationship-building}
Discuss the relationship-building stage in some detail; it's really important.
Relationships are a key part of cross-cultural research.  See Sue Smith, as 
Sean has mentioned.

This collaborative and cross-cultural approach has been and will be a 
significant influence on the direction of the research. (explore it more then)

%% This is a particularly important point
Without a high level of engagement, including introductions and logistical
assistance with travel permits, it would be impractical or impossible to conduct
such research for an Honours thesis.



\section{Qualitative and Interview Methods}

I rely on four sources of qualitative knowledge and context for the Yolngu
seasons, in order of importance:

\begin{itemize}
\item Interview with Yolngu people form the basis of my qualitative research, and
        are the definitive source of information about Yolngu seasons.
\item Interviews and discussion with non-Indigenous researchers or teachers experienced
        in remote communities help contextualise this knowledge, and warned of
        common misinterpretations - as well as pointing out nuance.
\item Published literature, particularly around cross-cultural research \citep[eg.][]{smith1999},
        Australian Indigenous seasons \citep[eg.][]{prober2011,oconnor2010}, and Yolngu
        seasons directly \citep{davis1989}.
\item Grey literature, such as posters produced for use in remote schools, workbooks
        for cross-cultural teacher training, etc.
\end{itemize}


I conducted a series of informal, semistructured interviews discussing
seasons and climate. Recruitment was primarily by a `snowball' pattern,
where participants were asked to suggest further potential participants.

The main `seeds' of this pattern were personal contacts, some Yolngu and some
non-Indigenous people who have spent decades living and working in remote
communities. I followed several disconnected threads such as the
\citet{CSIROcals} project, which helped to clarify and test my interpretation
of the interviews.

See \autoref{sec:ethics} for details of the human ethics approval.
Semi-structured interviews with Yolngu participants are my primary source
of information.  These interviews are utterly indispensible.
The literature contextualises and informs the interviews, and suggested many fruitful
directions for direct questions or later searches for documents.

The most consistently productive interviews were focussed on teaching about
seasons: names and definitions, typical conditions and timing, and touching on
extreme events and outliers. Thematic questions include:
\begin{itemize}
\item What are the names of the seasons?
\item When does [a season] usually occur?  How do you know when it starts (definition)?
\item How long does [a season] last?  What weather or events usually occur in this season?
\item Do you think the seasons have changed over your lifetime?  Why/why not?  How can you tell?
\item Are there some years where a season is skipped?  What happens?
\item Do you remember any unusual events?  What happened?
\item What might the calendar be like if [example climate impact] happened? 
      (eg changes to wind, temperature, rainfall patterns) 
\item What are some important ways you use the land which depend on seasons?
        Have these changed over time?  Why?
\end{itemize}

Following the interviews, I also engaged in substantial reflection on the nature
of my questions and the ambiguity of the responses and data I collected.
In this process, allowing participants to discuss whatever they felt relevant
is a key way to remain open to unexpected information - including insights
which changed my understanding of what a Yolngu seasonal calendar was!






\section{Bridging Approaches to Quantify a Calendar}


\paragraph{Observing seasonal onset}~\\
There is a considerable body of literature dealing with seasons defined
by observed environmental change - from temperature thresholds in
Scandinavia to 'leaf out' for botanists or butterflies for zoologists.

While such definitions are highly dependent on context, some common
elements can be observed.

Monotonic changes and qualitative changes are easiest to pin down.
Ie binary states, and/or only passing the threshold once per year.

TODO more on this, add references.

I've started reading the literature on historical climatology, but don't have 
enough depth to write much yet.  It's going to be an important angle though. 

Matching oral history with numerical data is an established technique in 
climate science, as recorded observations can greatly assist in reconstruction 
of conditions before the instrumental record began.  In this case, observations 
within living memory are an excellent source, as the instrumental record is 
even today very sparse in Arnhem Land.  Investigation of social adaptation to 
local climate is an interesting and often closely linked field.

Look more into non-observational records; as far as I know dendrology etc has 
not been done much in the region; this is relevant as a missing alternative to 
oral histories.



\paragraph{Weather observations} were generously supplied by the
Australian Bureau of Meterology.

\begin{table}[h]
    \centering
    \begin{tabular}{cllcccl}
        Station no.  &  Name                &  Location     &  Latitude   &  Longitude   &  Elevation  &  Opened   \\
        014401       &  WARRUWI AIRPORT     &  Warruwi      &  11.6500S   &  133.3797E   &  19.2       &  Jan 1916 \\
        014404       &  MILINGIMBI AIRPORT  &  Milingimbi   &  12.0932S   &  134.8919E   &  15.0       &  Mar 2003 \\
        014405       &  MANINGRIDA AIRPORT  &  Maningrida   &  12.0569S   &  134.2339E   &  28.1       &  Oct 2003 \\
        014508       &  GOVE AIRPORT        &  Nhulunbuy    &  12.2741S   &  136.8203E   &  51.6       &  Jan 1944 \\
        014517       &  NGAYAWILI           &  Galiwinku    &  11.9971S   &  135.5726E   &  08.1       &  Oct 1999
    \end{tabular}
    \caption[List of weather stations providing data]{
        Summary description of weather stations used.
        Locations are visible in \autoref{fig:arnhem-map}}
    \label{tab:weather-station-summary}
\end{table}

\autoref{tab:weather-station-summary} shows the name, ID, and location of
the weather stations used.
%
The weather variables of interest are:
\begin{itemize}
\item Rainfall in the 24 hours before 9am (local time), in milimeters.
\item Maximum temperature in the 24 hours after 9am (local time), in Degrees C.
\item Minimum temperature in the 24 hours before 9am (local time), in Degrees C.
\item Humidity measured as average daily dew point temperature, in Degrees C.
\item Wind speed measured in kilometers per hour, at 9am and 3pm local time.
\item Wind direction recorded as 16 compass points, at 9am and 3pm local time.
\end{itemize}

Participants suggested that season detection should not consider 3pm wind
due to the effect of the sea breeze, and that 6pm would be more suitable.
\autoref{fig:galiwinku-seabreeze-direction} showins wind direction by year
and day-of-year, as in \autoref{fig:galiwinku-observations}.
%
Similar charts for wind speed and other stations are included in the electronic
appendices.  These charts show a northerly sea breeze, generally strong in the
noon, 3pm, and 6pm data.  I use 3pm wind data, as these comments are outweighed
by the standard time and longer record at some stations.\\

The data are cleaned by discarding observations accumulated over multiple days.
Observations which have been quality-controlled by the BoM and are considered
`wrong', `suspect', or `inconsistent with other known information' are discarded.
Observations which have not been assessed are retained.

Missing data is not filled in any way, to convey the coverage of the record
and accurately represent missing data in the figures.  In scalar calculations,
missing data is propagated (eg ${NaN+10=NaN}$); in some aggregation
it is omitted from the sample (eg ${mean(1,NaN,5)=3}$, though
${mean(NaN,NaN)=NaN}$). Heatmap figures represent missing data as a black cell.


\paragraph{Seasons are detected} at daily resolution, matching the input
weather data.
Each of these methods assigns a confidence rating for each season to each day of
observations.  These ratings are then reconciled to a single season, or unknown,
for each day.  The reconciliation draws on participants' comments to drop inconsistent
ratings before comparing normalised season ratings.

For each season, I construct a set of boolean criteria based on participant
descriptions.  A raw index for each season is defined by
running these conditions over each day of data (giving True, False, or NaN),
and summing the values for each season for each day (giving NaN, 0, 1, ...).
\autoref{fig:season-definitions-code} shows the code used to build raw
seasonal indicies.

\begin{figure}[h]
    \lstinputlisting[firstline=31, lastline=55, xleftmargin=0.1\textwidth]{../code/season.py}
    \centering
    \caption[Python code: definition of season indicies]{
        Code listing of variables and conditions used to detect seasons.
        TODO - update detection, expand description, keep eye on line range.
        }
    \label{fig:season-definitions-code}
\end{figure}

Season-detection criteria can be defined in two ways: based on thresholds
in absolute measurement, or on trends.  I prefer the threshold approach,
as experimenting with trend-based detection did not yield reliable results.
While I consider changes in the weather a better fit for season detection,
and a strong area for further study, this approach is impractical given available
qualitative and quantitative data.

These thresholds are set based on \autoref{subsec:calendar-description};
for categorical criteria such as wind direction or the absence of rain days
I am confident that they are reasonably rigorous.  For quantitative criteria,
such as ``very hot'', ``humid'', rain on ``most days'' or ``less frequently'',
rigour is more dificult.

The ideal solution would be to find a historical record of what season
it was on various days; regression analysis, (un)supervised classification,
or machine learning techniques could then associate seasons with typical
conditions.  Unfortunately, no such record exists.  In searching for one
I spoke to Yolngu people, current and ex-mission workers, the Tropical
Rivers and Coastal Knowledge team at CSIRO, academics from Charles Darwin
University and Nungalinya College.

Instead, I tried a range of values and selected those which appeared to
best reproduce the described timing of each season in the Galiwinku and
Milingimbi data.  This selection was on the basis of normalised season
indicies before aggregation, so it tested for specific descriptions
rather than merely expected results.

This is probably the weakest step in the quantification method,
but there isn't a practical alternative.  TODO - explain mutation
testing shows (low, medium?) sensitivity to exact values -
calculate and show exact difference for electronic appendices

Finally, the index timeseries for each season is normalised by conversion to
a z-score, to allow direct comparison between seasons - the variation in data
completeness and number of conditions would not otherwise permit it.
This index is my representation of how well the season characterises that day's
conditions.  Aggregation of these indicies is covered in the next section.



\section{Quantitative Analysis}

All numeric tables and data visualisations in this thesis are produced
with Python scripts writen by the author.  The raw data supplied by the
Bureau of Meteorology are supplied in the electronic appendices, as
are the analysis scripts.  This makes all quantitative results trivial
to reproduce, and greatly facilitates replication for other calendars
or datasets.  Running the scripts requires Python version 3.5 or later,
as well as the Pandas (0.18+) and Seaborn (0.7+) libraries with their
dependencies.

~\\

TODO - the remainder of this section is waiting on finalisation of the
remaining results, particularly section numbers and calculation.

\autoref{subsec:detection-advice} shows that seasons may occur multiple
times in a given year, in any order.  This presents a significant
challenge to reliable detection of seasonal onset, as it rules out
the simple approach of detecting an event and forward-filling the
associated season to the next detected event.  Instead, I use a local
analysis approach which seperately analyses each day of data with
a small window only used for smoothing via a rolling mean of 3-7 days.

The cost of this well-grounded system is 



Emphasising onset timing
by asking ``when does [this season] typically begin?'' mischaracterises the
Yolngu calendar; a better formulation might be ``what season occurs most
often at this time of year?'' or ``what season best matches typical conditions
for each day of the year?''.  \autoref{subsec:results-season-detection}
explores several such approaches to characterising typical timing for seasons,
and discusses the benefits and drawbacks of each.


TODO - acknowledge importance of these tips and talk about relevance to
detection design.  Will end up in methods.  Most important bit is that
interruption and out-of-order means that forward-filling last known
doesn't work; have to actually check each day.

~\\

Calculation of summary statistics for each season has a firmer conceptual
grounding.  The daily data are grouped by detected season and summary statistics
for each subset calculated, including measures of central tendency and
variation in ... 
\begin{itemize}
\item Calculation of summary stats for each season – typical and variation in conditions, onset, etc
\item Comparison of stations, other data as inputs to season recognition.
\item Correlation to ENSO, monsoon, dipole moment, etc
\end{itemize}

Note that even the simplest of these is really novel, as nobody has
been able to calculate the summaries before!

