\chapter{Methods}
\label{ch:methods}

\section{Methodology}

My methodological approach is cross-disciplinary and exploratory.
%
The cross-disciplinary aspect is inherent in the topic; it would
be impossible to characterise Indigenous seasons in this way without
integrating qualitative and quantitative methods.
%
It is exploratory because there is no known precedent or published
method for quantification of an Australian Indigenous seasonal calendar.
This novelty means that the specific quantification methods used
were not known ahead of time, as they are informed and shaped by
interview data.  It also means that these methods themselves form
a novel result, which breaks ground for further research.\\

My specific methods fall into four distinct stages.
\begin{enumerate}
\item Draw on literature describing the Yolngu seasonal calendar
\item Interviews with Yolngu and non-Indigenous people who
    understand this calendar
\item Quantitative methods bridging literature and interview
    descriptions to the observational weather record, to characterise
    Yolngu seasons
\item Analysis of the derived seasonal record, such as typical
    conditions for each season and correlations with climate indicies.
\end{enumerate}
These stages reflect the cross-disciplinary or mixed-methods approach --
literature informing the direction of the research, and qualitative
methods providing the framework for well-founded numerical analysis.\\


\paragraph{Interview methodology} is particularly important when working
across cultures, for example when collaborating with Indigenous people
on western-style scientific research.

\todo{finish this paragraph}

Discuss the relationship-building stage in some detail; it's really important.
Relationships are a key part of cross-cultural research.  See Sue Smith, as
Sean has mentioned.

This collaborative and cross-cultural approach has been and will be a
significant influence on the direction of the research. (explore it more then)

%% This is a particularly important point
Without a high level of engagement, including introductions and logistical
assistance with travel permits, it would be impractical or impossible to
conduct such research for an Honours thesis.




\paragraph{Data scoping} is a key influence on methodology.  What data is
considered to be within scope or relvant to the research question?
What isn't?  How is the decision made?

\todo{outline the temporal scale I'm dealing with - recent
decades for both qualitative and numerical data.
Talk about practical aspect - both aquisition and analysis must be practical.
Consider relevance, robust analysis, reproducibility.
}


\paragraph{Detecting seasons} \todo{fill out this paragraph}



\paragraph{Quantitative} summaries are an important aspect of this
research, but on the outer edge of it's scope.  This analysis is therefore
limited, to the extent that it either informs judgement of earlier work
or demonstrates potential for further study in a particular area.

In practice, this means calculating a small number of summary statistics
-- those which are meaningful for a complex calendar and supported by the
data -- and proof-of-concept analysis of Yolngu seasons with respect to
a few climate indicies.



\section{Qualitative and Interview Methods}
\todo{Finish this section}


\subsection{Sample and Recruitment}
\todo{describe sample recruitment, size, demographics.
Note planned trip to Galiwinku fell through, explain why
and what fallback plan was.}


I rely on four sources of qualitative knowledge and context for the Yolngu
seasons, in order of importance:

\begin{itemize}
\item Interview with Yolngu people form the basis of my qualitative research, and
        are the definitive source of information about Yolngu seasons.
\item Interviews and discussion with non-Indigenous researchers or teachers experienced
        in remote communities help contextualise this knowledge, and warned of
        common misinterpretations - as well as pointing out nuance.
\item Published literature, particularly around cross-cultural research \citep[eg.][]{smith1999},
        Australian Indigenous seasons \citep[eg.][]{prober2011,oconnor2010}, and Yolngu
        seasons directly \citep{davis1989}.
\item Grey literature, such as posters produced for use in remote schools, workbooks
        for cross-cultural teacher training, etc.
\end{itemize}


I conducted a series of informal, semistructured interviews discussing
seasons and climate. Recruitment was primarily by a `snowball' pattern,
where participants were asked to suggest further potential participants.

The main `seeds' of this pattern were personal contacts, some Yolngu and some
non-Indigenous people who have spent decades living and working in remote
communities. I followed several disconnected threads such as the
\citet{CSIROcals} project, which helped to clarify and test my interpretation
of the interviews.

See \autoref{sec:ethics} for details of the human ethics approval.


\subsection{Interview Methods and Questions}
Semi-structured interviews with Yolngu participants are my primary source
of information.  These interviews are utterly indispensible.
The literature contextualises and informs the interviews, and suggested many fruitful
directions for direct questions or later searches for documents.

The most consistently productive interviews were focussed on teaching about
seasons: names and definitions, typical conditions and timing, and touching on
extreme events and outliers. Thematic questions include:
\begin{itemize}
\item What are the names of the seasons?
\item When does [a season] usually occur?  How do you know when it starts (definition)?
\item How long does [a season] last?  What weather or events usually occur in this season?
\item Do you think the seasons have changed over your lifetime?  Why/why not?  How can you tell?
\item Are there some years where a season is skipped?  What happens?
\item Do you remember any unusual events?  What happened?
\item What might the calendar be like if [example climate impact] happened?
      (eg changes to wind, temperature, rainfall patterns)
\item What are some important ways you use the land which depend on seasons?
        Have these changed over time?  Why?
\end{itemize}

Following the interviews, I also engaged in substantial reflection on the nature
of my questions and the ambiguity of the responses and data I collected.
In this process, allowing participants to discuss whatever they felt relevant
is a key way to remain open to unexpected information - including insights
which changed my understanding of what a Yolngu seasonal calendar was!






\section{Application: from Qualitative to Numerical Data}

This paper attempts to quantify only one Yolngu seasonal, with six seasons
each year defined by weather conditions and events.  See
\autoref{subsec:three-seasons-scales} for a brief explanation of other
Yolngu seasonal calendars, and an explanation of this focus.


\subsection{Weather Observations}

The numerical weather data were generously supplied by the Australian
Bureau of Meterology.  This section describes the basic numerical data,
quality checks, and handling of missing data.


\begin{table}[h]
    \centerline{
    \begin{tabular}{cllcccl}
        Station no.  &  Name                &  Location    &  Latitude  &  Longitude   &  Altitude  &  Opened   \\
        014401       &  Warruwi Airport     &  Warruwi     &  11.6500S  &  133.3797E   &  19m       &  Jan 1916 \\
        014404       &  Milingimbi Airport  &  Milingimbi  &  12.0932S  &  134.8919E   &  15m       &  Mar 2003 \\
        014405       &  Maningrida Airport  &  Maningrida  &  12.0569S  &  134.2339E   &  28m       &  Oct 2003 \\
        014508       &  Gove Airport        &  Nhulunbuy   &  12.2741S  &  136.8203E   &  52m       &  Jan 1944 \\
        014517       &  Ngayawili           &  Galiwinku   &  11.9971S  &  135.5726E   &  08m       &  Oct 1999
    \end{tabular}
    }
    \caption[List of weather stations providing data]{
        Summary description of weather stations used.
        Locations are visible in \autoref{fig:arnhem-map}}
    \label{tab:weather-station-summary}
\end{table}

\autoref{tab:weather-station-summary} shows the name, ID, and location of
the weather stations used.
%
The weather variables of interest are:
\begin{itemize}
\item Rainfall in the 24 hours before 9am (local time), in milimeters.
\item Maximum temperature in the 24 hours after 9am (local time), in Degrees C.
\item Minimum temperature in the 24 hours before 9am (local time), in Degrees C.
\item Humidity measured as average daily dew point temperature, in Degrees C.
\item Wind speed measured in kilometers per hour, at 9am and 3pm local time.
\item Wind direction recorded as 16 compass points, at 9am and 3pm local time.
\end{itemize}

Participants suggested that season detection should not consider 3pm wind
due to the effect of the sea breeze, and that 6pm would be more suitable.
\autoref{fig:galiwinku-seabreeze-direction} showins wind direction by year
and day-of-year, as in \autoref{fig:galiwinku-observations}.
%
Similar charts for wind speed and other stations are included in the electronic
appendices.  These charts show a northerly sea breeze, generally strong in the
noon, 3pm, and 6pm data.  I use 3pm wind data, as these comments are outweighed
by the standard time and longer record at some stations.\\

The data are cleaned by discarding observations accumulated over multiple days.
Observations which have been quality-controlled by the BoM and are considered
`wrong', `suspect', or `inconsistent with other known information' are discarded.
Observations which have not been assessed are retained.

Missing data is not filled in any way, to convey the coverage of the record
and accurately represent missing data in the figures.  In scalar calculations,
missing data is propagated (eg ${NaN+10=NaN}$); in some aggregation
it is omitted from the sample (eg ${mean(1,NaN,5)=3}$, though
${mean(NaN,NaN)=NaN}$). Heatmap figures represent missing data as a black cell.


\subsection{Detection of Seasons}

Seasons are detected at daily resolution, matching the input
weather data.
Each of these methods assigns a confidence rating for each season to each day of
observations.  These ratings are then reconciled to a single season, or unknown,
for each day.  The reconciliation draws on participants' comments to drop inconsistent
ratings before comparing normalised season ratings.

For each season, I construct a set of boolean criteria based on participant
descriptions.  A raw index for each season is defined by
running these conditions over each day of data (giving True, False, or NaN),
and summing the values for each season for each day (giving NaN, 0, 1, ...).
\autoref{fig:season-definitions-code} shows the code used to build raw
seasonal indicies.

Season-detection criteria can be defined in two ways: based on thresholds
in absolute measurement, or on trends.  I prefer the threshold approach,
as experimenting with trend-based detection did not yield reliable results.
While I consider changes in the weather a better fit for season detection,
and a strong area for further study, this approach is impractical given available
qualitative and quantitative data.

These thresholds are set based on \autoref{subsec:calendar-description};
for categorical criteria such as wind direction or the absence of rain days
I am confident that they are reasonably rigorous.  For quantitative criteria,
such as ``very hot'', ``humid'', rain on ``most days'' or ``less frequently'',
rigour is more dificult.

The ideal solution would be to find a historical record of what season
it was on various days; regression analysis, (un)supervised classification,
or machine learning techniques could then associate seasons with typical
conditions.  Unfortunately, no such record exists.  In searching for one
I spoke to Yolngu people, current and ex-mission workers, the Tropical
Rivers and Coastal Knowledge team at CSIRO, academics from Charles Darwin
University and Nungalinya College.

Instead, I tried a range of values and selected those which appeared to
best reproduce the described timing of each season in the Galiwinku and
Milingimbi data.  This selection was on the basis of normalised season
indicies before aggregation, so it tested for specific descriptions
rather than merely expected results.

This is probably the weakest step in the quantification method,
but there isn't a practical alternative.  \todo{explain mutation
testing shows (low, medium?) sensitivity to exact values -
calculate and show exact difference for electronic appendices}

Finally, the index timeseries for each season is normalised by conversion to
a z-score, to allow direct comparison between seasons - the variation in data
completeness and number of conditions would not otherwise permit it.
This index is my representation of how well the season characterises that day's
conditions.  Aggregation of these indicies is covered in the next section.



\section{Quantitative Analysis}

All numeric tables and data visualisations in this thesis are produced
with Python scripts writen by the author.  The raw data supplied by the
Bureau of Meteorology are supplied in the electronic appendices, as
are the analysis scripts.  This makes all quantitative results trivial
to reproduce, and greatly facilitates replication for other calendars
or datasets.  Running the scripts requires Python version 3.5 or later,
as well as the Pandas (0.18+) and Seaborn (0.7+) libraries with their
dependencies.

~\\

\autoref{subsec:detection-advice} shows that seasons may occur multiple
times in a given year, in any order.  This presents a significant
challenge to reliable detection of seasonal onset, as it rules out
the simple approach of detecting an event and forward-filling the
associated season to the next detected event.  Instead, I use a local
analysis approach which seperately analyses each day of data with
a small window only used for smoothing via a rolling mean of 3-7 days.

The cost of this well-grounded system is increased `jitter', rapid
switching between seasons in periods when more than one is a close
match for conditions.  To counter this, and account for the advice
that season change requires several days observation, I...\\

\todo{I have some ideas to counter this; eg boosting the index of
whatever prevailed on the previous day by 0.25 standard deviations.
Alternatively only change if season is same for two or three days.
Implement and see what shakes out.}

~\\

Yolngu seasons cannot be well characterised by onset timing.  Instead
of asking ``when does [this season] typically begin?'' - which implies
that each season begins once per year, at a roughly constant date -
it is better to ask which season best matches conditions at a date,
or which is most commonly observed.
%
\autoref{subsec:results-season-detection} explores several approaches
to characterising typical timing for seasons, and discusses the
benefits and drawbacks of each.

~\\

Calculation of summary statistics for each season has a firmer conceptual
grounding.  The daily data are grouped by detected season and summary statistics
for each subset calculated, including measures of central tendency and
variation in ...
\begin{itemize}
\item Calculation of summary stats for each season – typical and variation in conditions, onset, etc
\item Comparison of stations, other data as inputs to season recognition.
\item Correlation to ENSO, monsoon, dipole moment, etc
\end{itemize}

Note that even the simplest of these is really novel, as nobody has
been able to calculate the summaries before!

