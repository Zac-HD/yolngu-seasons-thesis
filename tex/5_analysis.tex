\chapter{Analysing Yolngu Seasons through Time}
\label{ch:analysis}

The purely quantitative methods, results, and interpretive discussion.

\section{Methods}

\begin{itemize}
\item Calculation of summary stats for each season – typical and variation in conditions, onset, etc
\item Comparison of stations, other data as inputs to season recognition.
\item Normalisation functions from observations to gridded data inc. model backcasts if applicable.
\item Correlation to ENSO, monsoon, dipole moment, etc
\item Changes over time (by decade?) – historical data
\item Changes over time – model outputs across models and scenarios
\end{itemize}

Note that even the simplest of these is really novel, as nobody has been able to calculate the summaries before!

The application element is conceptually simple:  based on the physical
descriptors and numerical data, quantitative characterisations of the
Yolngu seasons can be calculated.  These will include both typical values
and variation in season onset date, duration, temperatures, rainfall,
wind, and so on.

The exploration element will analyse the information generated for emergent
insights or other novel results.  The relationships between seasons, extreme
weather events and fire are of particular interest; along with the nature of
seasonal changes through time including scenarios under future climate change.
Exploratory findings may form a major result of the thesis.


\section{Results and Interpretation}
Outputs from the analyses above, with interpretation.

