\chapter{Abstract}

Indigenous seasonal calendars contain valuable traditional knowledge,
through their very definition:  where Western seasons are determined by
solar date and are invariant across continents,
indigenous seasons are defined by local conditions.
%
Indigenous seasons vary from place to place, are recognised by weather or ecological events,
take place over varying timescales, and occur at different times of year depending on environmental factors.

This thesis characterises a Yolngu seasonal calendar from Galiwinku,
on the north-east coast of Arnhem Land, in the Northern Territory of Australia.
%
The calendar is derived from a series of interviews with Yolngu people including elders, and
non-Indigenous teachers, mission workers, and scientists working in relevant fields.
%
I discuss my experience of ambiguity in the meaning of `season' and `calendar',
based on three distinctive seasonalities with different time-scales and definitions.
Of the three (monsoon, weather, and ecological) seasonalities discussed, weather-based
seasonality best supports mixed methods and cross-cultural understanding and is used throughout.

I then compare two methods for quantitative recognition of seasons based on weather records
and qualitative descriptions from research participants.  The first is objectively replicable,
deriving threshold values from the meteorological record for months typically associated with each season.
The second method is more precise, identifying onset in periods of rapid trend change in key variables.
This exploratory analysis results in a `quantified calendar' mapping Indigenous seasons to meteorology,
a result unprecedented in the surveyed literature.

Enabled by the quantified calendar I derive further novel results, including:
\begin{itemize}
\item typical values and variation in onset date, duration, and weather conditions for each season,
\item correlation between Yolgu seasons and climate cycles including
        the monsoon and El Nino oscillation,
\item decadal analysis of changes to Yolngu seasons over the historical record,
\item and, (as practical and if possible) discussion of what this might mean for Yolngu people and lifestyles.
\end{itemize}


