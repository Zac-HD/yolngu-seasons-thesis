\chapter{Introduction}
\label{ch:introduction}

Australia presents many challenges to good environmental management: its
climate is among the most variable on Earth, interleaving years of baking
droughts with flooding rainfall, the soils are relatively poor, and much of
the flora and fauna is endemic.
%
Since European colonisation, settlers have responded in many ways --
drawing on their traditons, science, and mateship among other strategies to cope
with harsh landscapes and extreme climates.
%
They have done well by some the standards -- the agricultural and natural
resource industries are large and stable \todo{REF}.  Other standards condemn
them: in pursuing these aims countless species have been driven to extinction,
whole ecosystems degraded, and natural wonders like the Great Barrier Reef
in serious danger -- damage that will take many generations to heal if it is
not already irreversible.

Australia as a nation could do better, if it learns from its mistakes and can
see the value in other perspectives.  Alternatives to an exploitative or
colonial approach are demonstrated by Indigenous Australian cultures, who
have lived in and with `country' -- not as a generalised or undifferentiated
type of place, but a complex, living entity made up of ``people, animals,
plants, Dreamings; underground, earth, soils, minerals and waters, surface
water, and air'' \citep{birdrose1996}.
%
To date, the second peoples of Australia (that is, non-Indigenous Australians)
have generally dissmissed, undervalued, and ignored Indigenous Knowledge
and perspectives; as recently as the 1960s schools taught that Indigenous
peoples and cultures offered nothing of value and would inevitably die out
\citep{flannery1994}.


More recently, Indigenous Knowledge is recognised and valued in an
increasing variety of contexts \citep[eg.][]{petheram2010,cochran2015,
berkes2012}.  One way to realise its value is to deepen our understanding
of natural cycles, especially as they may relate to land or natural resource
management.  This thesis seeks to bring Indigenous knowledge together with
western science, enriching both, in relation to a seasonal calender.
%
Seasonal calendars are constructed by humans as a usefully simplified
abstraction of annual variation in the environment - whether physical
changes such as temperature, or social events from birthdays to financial
reporting.  The definitions of seasons carry knowledge about annual cycles,
and also the domain for which the calendar is applied - eg. temperature
or taxation.

Documenting and quantifying seasons defined by biophysical conditions
or events allows researchers to draw on a significant body of existing,
implicit knowledge about the environment.
%
This knowledge supports decisions in a range of areas; from natural
resource management, agriculture, or energy demand forecasts, to
the timing of meetings and holidays or the materials used in high fashion \todo{insert refs for examples}.
Seasons have a profound impact on human societies, and understanding
them matters.  Knowledge of biophysical seasonality provides a nuanced
basis for assessments of climate change impacts, or the success of
adaptation or mitigation policies.  This will only become more important
as we move into the Anthropocene, and -- for good or ill -- humans assume
direct control over the environment.


Australia formally recognises four seasons, defined by Gregorian date:
Summer begins on December 1\textsuperscript{st}, Autumn on March
1\textsuperscript{st}, Winter on June 1\textsuperscript{st}, and Spring
on September 1\textsuperscript{st} \citep{wells2013}. This follows the
meteorological standard for seasons, rather
than the solar seasonal calendar marked by solsticies and equinoxes.
%
Throughout the temperate zone, seasons are informally recognised more by
temperature than date (Summer being the hottest and Winter the coldest time
of year), and thus varying from year to year.  While rainfall is often
strongly  seasonal the relationship varies from place to place, and it is
less recognised as an indicator.
%
However, temperate seasons are essentially irrelevant to the tropical and
equatorial climate of northern Australia.  In these areas a monsoonal Wet
and Dry season dominate, generally recognised by weather conditions.
but lacking any of the nuance or local knowledge embedded in Indigenous
calendars, which are defined by weather and ecological events.  These
seasons are typically recognised after the fact, and vary in timing and
length between years depending on the behaviour of their indicators.


An alternative way of defining seasonal cycles is by periodic ecological
events, whether plant (date of first flowers, etc) or animal (eg. migration).
The study of such events is called \textit{phenology}, which forms an
important complementary approach to meteorology in assessment of climate
change \citep[eg.][]{roy2000}.  \citet{menzel2006} analyse spring onset timing
at several sites across the United Kingdom and Germany, using dozens of
plant phenophases and butterfly appearance data, and find a strong climate
change signal in earlier onset and increased variability.  Such research
draws on written records, which generally begin in the 1700s for European
data or as early as the eighth century for Japanese observations \citep{sparks2002}.
%
Indigenous Ecological Knowledge (IEK) is derived from a significantly longer
tradition.  The ecological and phenological knowledge embedded
in Indigenous calendars may provide a rich complement to written records,
or an alternative in the many cases where such records do not exist \todo{REFS}.

Previous research on Australian Indigenous calendars \todo{example REFS}
has been qualitative, and focussed on the social or ecological aspects of
the seasons.  While important -- many communities are justly proud to share
their knowledge through CSIRO's poster project (see eg. \cref{fig:tiwi-seasons})
-- quantitative research on Indigenous seasons or even nuanced tropical seasons
is rare in Australia, and may be valuable in informing traditional or western
land and natural resource management.  This thesis will therefore treat
Indigenous knowledge as a \emph{framework for} research, not simply an
\emph{object of} research.


This research investigates Yolngu seasons in North-East Arnhem Land, with the
aim of developing an approach to integrating traditional seasonal knowledge
with a quantitative analytical approach to understanding the climate of a
place.  The Yolngu seasonal calendar is reasonably complex - see
\cref{subsec:three-seasons-scales} - with six major seasons defined by
weather and ecological events.  The exact definitions are highly localised,
and vary slightly even between communities sharing the same language.
%
The research questions that will be addressed in pursuing this aim are:
\begin{enumerate}
\item How are Yolngu seasons defined?  How do these definitions vary between
    people, places, and over time?
\item Can a Yolngu seasonal calendar be developed consistent with Indigenous
    understanding, and appropriate for cross-disciplinary analysis and interpretation?
\item What are the defining characteristics of this calendar, based on
    people's experiences?
\item What are the meteorological and climatological characteristics of this
    calendar?  How can it be defined by numerical analysis of weather observations?
\item What are the reccurring properties of this calendar?  Are there notable
    relationships with well-known climatological indicators?
\end{enumerate}

These questions follow the same four-step methodology as the project itself:
understand the system to ask meaningful questions; identify key
qualitative information; apply insights to construct exact definitions; and
finally conduct quantitative analysis followed by comparisons and interpretation.

The rarity of research integrating Indigenous Knowledge with physical sciences
necessitates a clear scope and statement of delimitations.  This study will
explore a single Indigenous calendar, with respect to recent weather data
and contempoary Yolngu understandings.  It will not attempt to address historical
or pre-colonial perspectives, linguistics or anthropology, or the implications
of climate change -- except as recommendations for further study.

The novelty and value of this work is in synthesis: bringing
togther Yolngu perspectives on the natural environment, and the
data-oriented approach of climate science.  A review of the literature
finds previous work on Indigenous seasons focused on qualitative description, but
no quantitative or numerical research (see \cref{sec:aus-indig-seasons}).
Such synthesis provides a crucial long-term perspective on
anthropogenic environmental change, and may produce novel and valuable
results for both scientific and indigenous stakeholders.



\subsection*{Thesis Structure}

The chapters of this thesis -- Introduction, Literature Review, Methods,
Results, Discussion, and Conclusion -- follow the usual convention for
a scientific report.  Within the Methods and Results chapters
the structure is slightly more complicated, due to the use of iterative and
cross-disciplinary approaches.  Quantitative methods are informed by
qualitative results, necessitating forward references.  Readers may wish
to read these chapters as paired sections, which reflects the research
process.

The introduction (this chapter) briefly introduces the topic of
indigenous seasons, the contribution of this research, context and
study area, and outlines the structure and scope of the thesis.
\nameref{ch:lit-review} (\cref{ch:lit-review}) reviews prior research
on indigenous seasons and relevant methods.  Note that other chapters
also reference literature where relevant.
\nameref{ch:methods} (\cref{ch:methods}) lays out the methods used in
qualitative research with Indigenous participants, quantitative meteorological
descriptions, and the methodological approach to synthesis.
\nameref{ch:results} (\cref{ch:results}) presents qualitative and
quantitative results, including interpretation of figures.
\nameref{ch:discussion} (\cref{ch:discussion}) interprets and discusses
all results.  It also reflects on methods used and created, novelty and
potential applications, and directions for further study.
\nameref{ch:conclusion} (\cref{ch:conclusion}) summarises the findings
and impact of the research.

The appendices (see \cref{ch:appendices}) are split into two parts.
The printed appendices are included with this document, and contain
important data such as additional figures or details of the Human Research Ethics
approval which would be disruptive in the main text.  The second part
is the electronic appendices, which contain all other material judged to
be useful for further study - including all numerical data and scripts
needed to reproduce the results.

