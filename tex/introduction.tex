\chapter{Introduction}
This chapter contains a brief introduction to australian indigenous seasonal calendars,
explains the focus, novelty, and delimitations of this research,
and lays out the structure of the thesis.


\section{Indigenous Seasons and Ecological Knowledge}
% - One paragraph on the global-scale reason to care about the topic
% - Lay out how indigenous seasons are defined (inc. international, contrast)
% - draw link to ecological knowledge
% - Explain shortcomings in existing research


This thesis investigates and characterises an indigenous seasonal calendar.
I combine Indigenous knowledge with Western climate science to characterise
local seasonality in NE Arnhem Land (see \autoref{fig:arnhem-map}),
as described by a Yolngu seasonal calendar I construct based on
interviews and supported by literature.


Where Australia's formally recognised seasons are defined by Gregorian date -
e.g. December 1 is by definition the first day of Summer - indigenous seasons
are defined by weather and ecological events.
%
This more closely matches intuitions about seasons as annually recurring
periods marked by weather - most commonly temperature, which tends to
follow sunlight intensity - than a date-based approach, though indigenous
seasons are more nuanced than ``hot == summer, cold == winter''.

Like the observation-based approach used in some Nordic nations, where
seasons are defined by persistent temperature change (REF),
Indigenous Australian seasons are recognised after the fact and vary
between years depending on the behaviour of their indicators.


The Yolngu seasonal calendar was chosen as the case study for this thesis
as personal relationships in the region facilitated qualitative research,
and the emphasis on meteorological indicators allows quantification based
on the observational weather record.
    


~\\
~\\
Section beyond this point is DRAFT NOTES ONLY


Highly localised.
Yolngu calendars of North-East Arnhem Land divide the year into
six seasons based on prevailing wind, rainfall, temperature,
and a variety of other factors.

Derivation of the seasonal calendar from current conditions,
along with the impossibility of applying the European seasonal calendar
to the tropical seasonality, make Yolngu seasons an excellent case study.

See \citet{menzel2006} re modern scientific use of ecological changes
as indicators of seasonal onset or climate change.



Indigenous knowledge is recognised and valued in an increasing variety
of contexts \citep[eg.][]{petheram2010,cochran2015,berkes2012} –
but integration with the physical sciences is still relatively rare and ad-hoc.
%
I argue that such synthesis provides a crucial long-term perspective on
anthropogenic environmental change, and demonstrate that such synthesis
can produce novel results for both scientific and indigenous stakeholders.

Ecological knowledge is embedded in calendars.



\section{Delimitations and novelty}
% May want to present novelty, then delimitations...

NEEDS PARA ON DELIMITATIONS - see and work from mid-course review feedback to inform this.

~\\

Existing research on indigenous calendars has focussed on the social or ecological aspects,
documenting indigenous knowledge – see eg \textit{Man of All Seasons} \citep{davis1989},
the Indigenous Weather Knowledge project \citet{BOM-iwk},
or CSIRO’s Tropical Rivers and Coastal Knowledge (TRaCK) program \citep{CSIROcals,oconnor2010}.

There is however a paucity of research which treats indigenous knowledge
as a \emph{framework for} research, instead of an \emph{object of} research.

This trend manifests as a gap in the literature (see \autoref{sec:lit-review}),
despite the novely and potential value of combining Indigenous seasonal knowledge
with quantitative climate science.

PARA ON LIT SEARCH STRATEGY

None of the participants I interviewed were aware of attempts to quantify indigenous seasons,
and several made spontaneous comments as to the novelty of this approach.





\section{Structure of the Thesis}

This document is structured with a variation of the standard format for scientific reports,
where the body chapters for a literature review, methods, and results
have been organised by topic rather than content.

The aim of this format is present the four logical stages of my research
in ordered and reasonably self-contained chapters.
It also reflects the overarching methodology of distinct steps that are
later extended.  The academic and local context provides direction and focus
for the qualitative research.  In turn, the qualitative data and context
allows construction of a bridge to quantitative data and methods.


\paragraph{Context:}
\autoref{ch:context}, \textit{\nameref{ch:context}} contextualises the research.
It covers the basic context of the research,
regarding the location and past studies of the Yolngu calendar;
finds a gap in the literature and surveys relevant fields;
and lays out the overarching methodology of and strategy for this research.

Note that while the overarching methodology and general review of the relevant literature
are contained in this section, each of the body chapters describes the applicable methods
and references more specific literature as appropriate.


\paragraph{Qualitative:}
\autoref{ch:seasons}, \textit{\nameref{ch:seasons}} focusses on the qualitative research component.
It discusses interview methods, considerations for indigenous knowledge, and integration with literature.
The primary results are a reflection on process and outcomes,
and a qualitative description of Yolngu seasons.


\paragraph{Bridging:}
\autoref{ch:quantify}, \textit{\nameref{ch:quantify}} shows the creation of a quantified Yolngu calendar.


\paragraph{Quantitative:}
\autoref{ch:analysis}, \textit{\nameref{ch:analysis}} shows the analysis of the quantified calendar.

