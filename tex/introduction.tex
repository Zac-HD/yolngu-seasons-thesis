\chapter{Introduction}
\label{ch:introduction}

% Write intro starting with wider context - the wider issue is that Australia is a challenging environment (from scratch)
% Then - we're not managing that well enough - indigenous understandings can be valuable (some below)
% Then - looking at seasons and climate - why - what's the existing research (much below)
% Then - specific aims and questions


This thesis will investigate and characterise an Indigenous seasonal calendar.
I combine Indigenous knowledge with Western climate science to characterise
local seasonality in NE Arnhem Land (see \cref{fig:arnhem-map}),
as described by a Yolngu seasonal calendar I construct based on
interviews and supported by literature.

Previous research \todo{REFS} has focussed on the social or ecological aspects of Australian
Indigenous calendars.  This thesis instead treats Indigenous knowledge as a
\emph{framework for} research, not simply an \emph{object of} research.
Togther with an cross-disciplinary approach quantifying Indigenous seasons,
my results are highly novel -- and indeed, to my knowledge unique.


Seasonal calendars are constructed by humans as a usefully simplified
abstraction of annual variation in the environment - whether physical
changes such as temperature, or social events from birthdays to financial
reporting.  The definitions of seasons carry knowledge about annual cycles,
and also the domain for which the calendar is applied - eg. temperature
or taxation.
%
Documenting and quantifying seasons defined by biophysical conditions
or events allows researchers to draw on a significant body of existing,
implicit knowledge about the environment.

This knowledge supports decisions in a range of areas; from natural
resource management, agriculture, or energy demand forecasts, to
the timing of meetings and holidays or the materials used in high fashion \todo{insert refs for examples}.
Seasons have a profound impact on human societies, and understanding
them matters.  Knowledge of biophysical seasonality provides a nuanced
basis for assessments of climate change impacts, or the success of
adaptation or mitigation policies.  This will only become more important
as we move into the Anthropocene, and -- for good or ill -- humans assume
direct control over the environment.


Australia formally recognises four seasons, defined by Gregorian date:
Summer begins on December 1\textsuperscript{st}, Autumn on March
1\textsuperscript{st}, Winter on June 1\textsuperscript{st}, and Spring
on September 1\textsuperscript{st} \citep{wells2013}. This follows the
meteorological standard for seasons, originating in Europe, rather
than the solar seasonal calendar marked by solsticies and equinoxes.
%
Throughout the temperate zone, seasons are informally recognised more by
temperature than date (Summer being the hottest and Winter the coldest time
of year), and thus varying from year to year.  While rainfall is often
strongly  seasonal the relationship varies from place to place, and it is
less recognised as an indicator.
%
However, temperate seasons are essentially irrelevant to the tropical and
equatorial climate of northern Australia.  In these areas a monsoonal Wet
and Dry season dominate, generally recognised by weather conditions.
This system is highly simplistic, lacking any of the nuance or local
knowledge embedded in Indigenous calendars.  Australia would be well
served by a more modern view than this European inheritance.


An alternative way of defining seasonal cycles is by periodic ecological
events, whether plant (date of first flowers, etc) or animal (eg. migration).
The study of such events is called \textit{phenology}, which forms an
important complementary approach to meteorology in assessment of climate
change \citep[eg.][]{roy2000}.  \citet{menzel2006} analyse spring onset timing
at several sites across the United Kingdom and Germany, using dozens of
plant phenophases and butterfly appearance data, and find a strong climate
change signal in earlier onset and increased variability.  Such research
draws on written records, which generally begin in the 1700s for European
data or as early as the eighth century for Japanese observations \citep{sparks2002}.

Indigenous Ecological Knowledge (IEK) is derived from a significantly longer
tradition.  The ecological and phenological knowledge embedded
in Indigenous calendars may provide a rich complement to written records,
or an alternative in the many cases where such records do not exist \todo{REFS}.


Australian Indigenous seasons are defined by weather and ecological events.
They are typically recognised after the fact, and vary in timing and length
between years depending on the behaviour of their indicators.  This study
focusses on the seasonal calendar of the Yolngu people, in Arnhem Land (see
\cref{fig:arnhem-map}).
%
The Yolngu seasonal calendar is reasonably complex - see
\cref{subsec:three-seasons-scales} - but there are six major seasons
analysed below, defined by prevailing wind, rainfall, temperature, and a
variety of other factors.  The exact definitions are highly localised,
and vary slightly even between communities sharing the same language.



%%%%%%%%%      Scope and Contribution       %%%%%%%%%%%

This study will explore Yolngu seasons, from North-East Arnhem Land.
The case study will focus on the towns of Galiwinku and Milingimbi,
for several practical reasons.  Personal connections and logistical support
from the Uniting Church greatly facilitate qualitative research.
Yolngu seasons emphasise meteorological indicators, which will allow principled
quantification based on the observational weather record.  Finally, research
on Indigenous seasons or even nuanced tropical seasons is rare in Australia,
and may be valuable in informing traditional or western land and natural
resource management.



Indigenous knowledge is recognised and valued in an increasing variety
of contexts \citep[eg.][]{petheram2010,cochran2015,berkes2012} -
but integration with the physical sciences is still relatively rare and
\textit{ad hoc}.  Work in this novel area would bear substantially more study than
is possible within the time and resource constraints of an Honours thesis,
making a clear scope particularly important.
%
Three important limits are to consider the calendar of only one Indigenous
language group, consider recent data and understandings only (avoiding
investigation of past or future change), and avoiding overly complex
analysis in a poorly understood field.

This research investigates Yolngu seasons in North-East Arnhem Land, with the
aim of developing an approach to integrating traditional seasonal knowledge
with a quantitative analytical approach to understanding the climate of a
place.
%
The research questions that will be addressed in pursuing this aim are:
\begin{enumerate}
\item How are Yolngu seasons defined?  How do these definitions vary between
    people, places, and over time?
\item Can a Yolngu seasonal calendar be developed consistent with Indigenous
    understanding, and appropriate for cross-disciplinary analysis and interpretation?
\item What are the defining characteristics of this calendar, based on
    people's experiences?
\item What are the meteorological and climatological characteristics of this
    calendar?  How can it be defined by numerical analysis of weather observations?
\item What are the reccurring properties of this calendar?  Are there notable
    relationships with well-known climatological indicators?
\end{enumerate}


%%%%%%%     basically the scoping and delimitations section     %%%%%%%%

These questions follow the same four-step methodology as the project itself:
understand the system to ask meaningful questions; identify key
qualitative information; apply insights to construct exact definitions; and
finally conduct quantitative analysis. followed by comparisons and interpretation.
%
This research is not a deep or rigorous exploration of historical Yolngu
perspectives; all interviews were conducted in English and in Darwin, and
the small number of participants limits investigation of the social or
cultural aspects of Indigenous Knowledge.  The climatological analysis
does not consider trend change, or more than basic correlation with
wider climatic forcings.

Instead, the novelty and value of this work is in synthesis: bringing
togther Yolngu perspectives on the natural environment, and the
data-oriented approach of climate science.  A review of the literature
finds previous work on Indigenous seasons focused on qualitative description, but
no quantitative or numerical research (see \cref{sec:aus-indig-seasons}).
Such synthesis provides a crucial long-term perspective on
anthropogenic environmental change, and may produce novel and valuable
results for both scientific and indigenous stakeholders.

The qualitative data describes the
definitions of seasons and conditions in recent decades; the quantitative
weather record begins with the installation of automatic weather stations
at several airports between 1990 and 2003  (some earlier observations exist,
but do not include all required variables).  The results do not support
speculatation on the past or future state of Yolngu seasons.  The methods
however are applicable to data such as climate models, which provide
a clear direction for further work.



\subsection*{Thesis Structure}

The chapters of this thesis -- Introduction, Literature Review, Methods,
Results, Discussion, and Conclusion -- follow the usual convention for
a scientific report.  Within the Methods and Results chapters
the structure is slightly more complicated, due to the use of iterative and
cross-disciplinary approaches.  Quantitative methods are informed by
qualitative results, necessitating forward references.  Readers may wish
to read these chapters as paired sections, which reflects the research
process.


The introduction (this chapter) briefly introduces the topic of
indigenous seasons, the contribution of this research, context and
study area, and outlines the structure and scope of the thesis.
%
\nameref{ch:lit-review} (\cref{ch:lit-review}) reviews prior research
on indigenous seasons and relevant methods.  Note that other chapters
also reference literature where relevant.
%
\nameref{ch:methods} (\cref{ch:methods}) lays out the methods used in
qualitative research with Indigenous participants, quantitative meteorological
descriptions, and the methodological approach to synthesis.

\nameref{ch:results} (\cref{ch:results}) presents qualitative and
quantitative results, including interpretation of figures.
%
\nameref{ch:discussion} (\cref{ch:discussion}) interprets and discusses
all results.  It also reflects on methods used and created, novelty and
potential applications, and directions for further study.
%
\nameref{ch:conclusion} (\cref{ch:conclusion}) summarises the findings
and impact of the research.

The appendices (see \cref{ch:appendices}) are split into two parts.
The printed appendices are included with this document, and contain
important data such as additional figures or details of the Human Research Ethics
approval which would be disruptive in the main text.  The second part
is the electronic appendices, which contain all other material judged to
be useful for further study - including all numerical data and scripts
needed to reproduce the results.

