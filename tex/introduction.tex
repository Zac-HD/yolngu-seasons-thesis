\chapter{Introduction}
\label{ch:introduction}

Australia presents many challenges to good environmental management: its
climate is among the most variable on Earth, combining years of baking
droughts with flooding rainfall, the soils are relatively poor, and much of
the flora and fauna is endemic or endangered \citep{davies1999}.
%
Since European colonisation, settlers have responded in many ways --
drawing on their traditons, science, and social bonds among other strategies to cope
with harsh landscapes and extreme climates.
%
They have done well by some standards -- agricultural and natural
resource industries are large and stable.  By other standards they have not:
their practices and strategies have driven countless species to extinction,
degraded ecosystems, and placed natural wonders like the Great Barrier Reef
in serious danger \citep{flannery1994}.  Much of this damage to the common heritage of Australians
is irreversible or will take many generations to heal.

Australian people and institutions could learn from their colonial and exploitative mistakes,
and appreciate the value in other perspectives.  Alternatives are demonstrated
by Indigenous Australian cultures, who have lived in and with `country' for
thousands of years -- not as a generalised or undifferentiated
type of place, but a complex, living entity made up of ``people, animals,
plants, Dreamings; underground, earth, soils, minerals and waters, surface
water, and air'' \citep[][p7]{birdrose1996}.
%
To date, non-Indigenous Australians
have all too often dissmissed, ignored, and undervalued Indigenous knowledge
and perspectives.  As recently as the 1960s schools taught that Indigenous
people and cultures offered nothing of value and would inevitably die out
\citep{flannery1994}.


More recently, Indigenous knowledges and knowledge systems have become recognised and valued in an
increasing variety of contexts \citep[eg.][]{petheram2010,cochran2015,
berkes2012}.  One way to realise this value is to deepen our understanding
of natural cycles, especially as they may relate to land or natural resource
management.

%\section*{Understanding Seasonal Calendars}

This thesis seeks to bring Indigenous knowledge together with
western science, enriching both, in relation to a seasonal calender.
Seasonal calendars are constructed by humans as a useful simplification of
annual variation and cycles, coupled with domain knowledge -- whether
physical changes such as temperature, or social and economic events from birthdays to
financial reporting.  Effective land and natural resource management relies on
local environmental knowledge of this kind.

Documenting and quantifying seasons defined by biophysical conditions
or events allows researchers to draw on a significant body of existing,
implicit knowledge about the environment.
%
This knowledge supports decisions in a range of areas; from natural
resource management to agriculture \citep{woodward2012a,ens2012}, to
energy demand forecasts, and the timing of meetings or holidays.
Seasons have a profound impact on human societies, and understanding
them matters.  Knowledge of biophysical seasonality provides a nuanced
basis for assessments of climate change impacts, or the success of
adaptation or mitigation policies \citep{green2010a,stevenson1996,prober2011}.
This will only become more important as we move further into the Anthropocene
\citep{steffen2007}, and -- for good or ill -- humans exert increasing
influence over the global environment.


Australian governments formally recognise four seasons, defined by Gregorian date:
Summer begins on December 1\textsuperscript{st}, Autumn on March
1\textsuperscript{st}, Winter on June 1\textsuperscript{st}, and Spring
on September 1\textsuperscript{st} \citep{wells2013}. This follows the
meteorological standard for seasons, rather
than the solar seasonal calendar marked by solsticies and equinoxes.
%
Throughout the temperate zone, seasons are informally recognised more by
temperature than date (summer being the hottest and winter the coldest time
of year), and thus varying from year to year.  While rainfall is often
strongly seasonal, the relationship varies from place to place and it is
less recognised as an indicator.
%
However, temperate seasons are essentially irrelevant to the tropical and
equatorial climate of northern Australia.  In these areas a monsoonal Wet
and Dry season dominate, generally recognised by weather conditions \citep[eg.][]{kingsley2003,willmett2009}
but lacking any of the nuance or local knowledge embedded in Indigenous
calendars, which are defined by weather and ecological events.  Indigenous
seasons may vary in timing and length between years as their indicators vary.
Seasonal onset is declared retrospectively, some days after the change.


An alternative way of defining seasonal cycles is by periodic ecological
events, whether plant (date of first flowers, etc) or animal (eg. migration).
The study of such events is called \textit{phenology}, which forms an
important complementary approach to meteorology in assessment of climate
change \citep[eg.][]{roy2000}.  \citet{menzel2006} analyse spring onset timing
at several sites across the United Kingdom and Germany, using dozens of
plant phenophases and butterfly appearance data, and find a strong climate
change signal in earlier onset and increased variability.  Such research
may also draw on written records, which generally begin in the 1700s for European
data and as early as 1180 in China \citep[][p97]{yoshino1996} or the eighth century
for Japanese observations \citep{sparks2002}.
%
Indigenous knowledge is derived from a significantly longer
tradition.  The ecological and phenological knowledge embedded
in Indigenous calendars may provide a rich complement to written records,
or an alternative in the many cases where such records do not exist.

\defcitealias{BOM-iwk}{BoM 2016}

Previous research on Australian Indigenous calendars
\citep[eg. BoM 2016,][]{CSIROcals,clarke2009,davis1989,atlas2014}
has been qualitative, and focussed on the social or ecological aspects of
the seasons.  Many communities are justifiably proud to share
their knowledge through CSIRO's poster project (see eg. \cref{fig:tiwi-seasons}).
Quantitative research on Indigenous seasons or even nuanced tropical seasons
is rare in Australia, and may be valuable in informing traditional or western
land and natural resource management.  This thesis will therefore treat
Indigenous knowledge as a \emph{framework for} research, not simply an
\emph{object of} research.

%\section*{Aim and Research Questions}

This research investigates Yolngu seasons in North-East Arnhem Land, with the
aim of developing an approach to integrating traditional seasonal knowledge
with a quantitative analytical approach, in order to better understand the climate
of a place.  Yolngu seasonal calendars are complex -- see
\cref{subsec:three-seasons-scales} -- with five \citep{davis1989}, six
\citep{atlas2014}, or seven \citep{barber2005} seasons defined by weather and
ecological events depending on location.  The exact definitions are highly localised,
and vary even between communities sharing the same language.

The project adopts a four-step methodology:  understand the system to ask
meaningful questions; identify key qualitative information; apply insights to
construct exact definitions; and finally conduct quantitative analysis
followed by comparisons and interpretation.  Within this framework, it
addresses the following research questions:
\begin{enumerate}
\item How are Yolngu seasons and calendar defined, and what are the definitions for each season?
\item How do these definitions of seasons vary between people or places?
\item Can a Yolngu seasonal calendar be developed which is consistent with Indigenous
    understanding and appropriate for cross-disciplinary analysis and interpretation?
\item What are the defining experiential, meteorological, and climatological
    characteristics of this calendar?
\item How can this calendar be defined by and applied in numerical analysis of weather observations?
\end{enumerate}

Because research integrating Indigenous Knowledge with physical sciences is
unusual, it is important to articulate a clear scope and statement of delimitations.
This study will explore a single Indigenous calendar, with respect to recent weather data
and contempoary Yolngu understandings.  It will not attempt to address historical
or pre-colonial perspectives, linguistics, or the implications
of climate change -- except as recommendations for further study.

The novelty and value of this work is in synthesis, bringing
togther Yolngu perspectives on the natural environment, and the
data-oriented approach of climate science.  A review of the literature
finds previous work on Indigenous seasons focused on qualitative description, but
no quantitative or numerical research (see \cref{sec:aus-indig-seasons}).
Such synthesis provides a crucial long-term perspective on
anthropogenic environmental change, and may produce novel and valuable
results for both scientific and Indigenous stakeholders.


\clearpage
\subsection*{Thesis Structure}

The chapters of this thesis follow the usual convention for a scientific
report: Introduction, Literature Review, Methods, Results, Discussion,
and Conclusion.
%
Use of iterative methods makes the structure of the Methods and Results
chapters slightly more complicated.  These chapters can be read by sections,
coming back to the second-stage methods after reading first-stage results,
a pattern which reflects the research process.

This chapter (the Introduction) briefly introduces the topic of
Indigenous seasons, the contribution of this research, context and
study area, and outlines the structure and scope of the thesis.
The \nameref{ch:lit-review} (\cref{ch:lit-review}) reviews prior research
on Indigenous knowledge and seasons, tropical seasonality in northern
Australia, and qualitative research methods.
%
\cref{ch:methods} lays out the methods used in the qualitative research
with Indigenous participants, quantitative meteorological descriptions,
and the methodological approach to synthesis.  \cref{ch:results} presents
qualitative and quantitative results, whih are then discussed and interpreted
in \cref{ch:discussion}.  This chapter also includes reflection on the
methods, novelty, and potential implications of the results.
\cref{ch:conclusion} summarises the findings and impact of the research,
and proposes directions for further study.

The appendices are split into two parts.  The printed appendices,
\cref{ch:appendices} at the end of this document, contain additional figures
which would interrupt the flow of the main text.  The second part
is the electronic appendices, which contain all other material judged to
be useful for further study - including all numerical data and original computer code
needed to reproduce the results.

