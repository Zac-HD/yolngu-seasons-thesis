\chapter{Creating a Quantified Yolngu Calendar}
\label{ch:quantify}

This chapter forms the bridge from purely qualitative to purely quantitative
methods.  I discussed quantification with Yolngu participants to elicit
key details, and draw comparisons to observational season detection from
the literature.

Based on daily weather observations provided by the \citet{BOM-data}
(most of which are freely available online), I characterise
the Yolngu seasons calendar presented in \autoref{ch:seasons}
using two statistical approaches.


\section{Methods}

Data analysis and presentation is entirely via my own Python scripts,
using a variety of scientific and data-analysis libraries.
I lean heavily on Pandas, Seaborn, NumPy, and Matplotlib.
For details and to read the -- well commented -- code, see
\autoref{sec:appendix-code}


\paragraph{Observing seasonal onset}~\\
There is a considerable body of literature dealing with seasons defined
by observed environmental change - from temperature thresholds in
Scandinavia to 'leaf out' for botanists or butterflies for zoologists.

While such definitions are highly dependent on context, some common
elements can be observed.  

Monotonic changes and qualitative changes are easiest to pin down.
Ie binary states, and/or only passing the threshold once per year.

TODO more on this, add references.


\paragraph{Yolngu participants} made several comments which are not about
the seasonal calendar itself, but are of vital importance when quantifying
it - teaching about how exactly to recognise each season.

Seasons are not limited to a particular order, and may happen one or more
times per year (by definition zero or more; historically at least once).
Any season may be `interrupted' by another if the latter's characteristic
conditions hold for longer than two or three days.
Where non-indigenous people might discuss a `cold snap', Yolngu might
say that winter had interrupted summer for a few days!

Wind plays a defining role in Yolngu seasons, but not all winds are
seasonal indicators.  In coastal communities, the day-time and especially
afternoon wind is dominated by the sea breeze, and carries relatively little
seasonal information.  Participants recommended considering wind at 9am
and 6pm, saying that the meteorological standard of 3pm would not be useful.

TODO check notes for other issues.


\paragraph{Weather observations} were generously supplied by the
Australian Bureau of Meterology.

\begin{table}[h]
    \centering
    \begin{tabular}{llllll}
        Station no.  &  Name                 &  Location     &  Latitude   &  Longitude   &  Elevation \\
%
        014401       &  WARRUWI AIRPORT      &  Warruwi      &  11.6500S   &  133.3797E   &  19.2      \\
        014404       &  MILINGIMBI AIRPORT   &  Milingimbi   &  12.0932S   &  134.8919E   &  15.0      \\
        014405       &  MANINGRIDA AIRPORT   &  Maningrida   &  12.0569S   &  134.2339E   &  28.1      \\
        014508       &  GOVE AIRPORT         &  Nhulunbuy    &  12.2741S   &  136.8203E   &  51.6      \\
        014517       &  NGAYAWILI            &  Galiwinku    &  11.9971S   &  135.5726E   &  8.1
    \end{tabular}
    \caption{Summary description of weather stations used.
        Locations are visible in \autoref{fig:arnhem-map}}
    \label{tab:weather-station-summary}
\end{table}

\autoref{tab:weather-station-summary} shows the name, ID, and location of
the weather stations used.

The weather variables of interest are:
\begin{itemize}
\item Rainfall in the 24 hours before 9am (local time), in milimeters.
\item Maximum temperature in the 24 hours after 9am (local time), in Degrees C.
\item Minimum temperature in the 24 hours before 9am (local time), in Degrees C.
\item Humidity measured as average daily dew point temperature, in Degrees C.
\item Wind speed measured in kilometers per hour, at 9am and 6pm local time.
\item Wind direction recorded as 16 compass points, at 9am and 6pm local time.
\end{itemize}

The data are cleaned by discarding observations accumulated over multiple days.
Observations which have been quality-controlled by the BoM and are considered
`wrong', `suspect', or `inconsistent with other known information' are discarded.
Observations which have not been assessed are retained.


\paragraph{Statistical approaches}~\\
I use two statistical approaches to detection of seasonal onset.

The first is to calculate threshold levels, chosen based on typical conditions
in months which were mentioned as typical for each season in my interviews.
TODO - add detail on how to determine typical - likley to be a percentile or something.

The second is to calculate the local trend in each varaible, identifying seasonal
change when on the normalised rate of change is high and the sign is correct.
TODO - more details on this too.

Each of these methods assigns a confidence rating for each season to each day of
observations.  These ratings are then reconciled to a single season, or unknown,
for each day.  The reconciliation draws on participants' comments to drop inconsistent
ratings before comparing normalised season ratings.




\section{Results and Discussion}

\subsection{Summary of a Yolngu Calendar}
Following \autoref{ch:seasons}, \autoref{tab:quant-seasons-summary} shows
typical timing and recognition criteria for each season.


\begin{table}[h]
    \centering
    \begin{tabular}{llllll}
        Season          &  Typical Months       &  Criteria to recognise                    \\
        Dhuludur        &  Oct, Nov, Dec        &  Cool at night, mixed wind, first rain    \\
        Barramirri      &  Dec, Jan, Feb        &  NW wind, heavy rain most days            \\
        Mayaltha        &  Feb, Mar             &  NW wind, approx weekly rain              \\
        Midawarr        &  Mar, Apr, May        &  NE to E wind, less rain, last storm      \\
        Dharrathamirri  &  May, Jun, Jul, Aug   &  No rain, consistent ESE-SSE wind         \\
        Rarrandharr     &  Sep, Oct             &  Hot days, low humidity                   
    \end{tabular}
    \caption{A quantifiable summary of the Yolngu seasons case study.}
    \label{tab:quant-seasons-summary}
\end{table}


Additionally, some participants offered tips for recognising seasons,
or information about seasonal patterns.

\begin{itemize}
\item The onset timing of seasons varies between years, with weather conditions. 

\item Seasons can `interrupt' one another, if conditions oscillate appropriately.
        It is however important to avoid premature declarations, so at least two to
        three days of the interrupting season's conditions should be observed.

\item Consequently, seasons may begin and end zero or more times each year -
        and at least once in every annual cycle to date.

\item Seasons are not required to begin in the standard order, though most do.

\item At Galiwinku, and presumably in other coastal locations, the afternoon
        sea breeze obscures the seasonal signal in wind direction.
        9am wind is fine.  3pm is too early.  At various times, 5pm-9pm is best.
        (The analysis below uses 6pm wind for all evening observations.)
\end{itemize}



\subsection{The Observational Weather Record in NE Arnhem Land}
TODO - tables etc. if I need them; discuss with Janette

Figures \ref{fig:galiwinku-observations} and \ref{fig:milingimbi-observations}
show historical weather observations at Galiwinku and Milingimbi as
sets of heatmaps, where for each variable a grid of year (y) and day-of-year(x)
is shaded according to the observation on that date.
Black cells indicate missing data.
In each figure, the variation in seasonality between years is clearly visible.


\begin{figure}[p]
    \centering
    \includegraphics[width=\textwidth]{galiwinku-all.pdf}
    \caption[Historical weather observations at Elcho Island]{
        Historical weather observations at Elcho Island.
        Each panel shows a single variable, with years on the y axis and day-of-year on the x.
        Note that each variable has a different seasonal pattern,
        and that seasonality changes from year to year.}
    \label{fig:galiwinku-observations}
\end{figure}
% Note - if possible, these figures should be on facing pages to allow easy comparison
\begin{figure}[p]
    \centering
    \includegraphics[width=\textwidth]{milingimbi-all.pdf}
    \caption[Historical weather observations at Milingimbi Airport]{
        Historical weather observations at Milingimbi Airport.
        Each panel shows a single variable, with years on the y axis and day-of-year on the x.
        Note that each variable has a different seasonal pattern,
        and that seasonality changes from year to year.}
    \label{fig:milingimbi-observations}
\end{figure}


\autoref{tab:galiwinku-monthly-summary} shows average conditions for each
month at Galiwinku.

\begin{landscape}
\begin{table}
    \input{../output/galiwinku-monthly-summary.tex}
    \caption{Monthly weather observations at Galiwinku.
        Numerical data aggregated by mean, wind direction by mode.}
    \label{tab:galiwinku-monthly-summary}
\end{table}
\end{landscape}



\subsection{Detection of Seasons}
For each season, run detection by the approaches given above.
Present as a multipanel for seach approach, with a panel for each season
and a final full-colour panel for all seasons.  Only provide this for the
Galiwinku data; others can stick to the electronic appendicies.

Discuss briefly which method is considered superior and why; note
that \autoref{ch:analysis} will, to be concise, stick to that one.
The other \emph{may} be used to reproduce results in an appendix.

Close chapter with a table (or similar) presenting seasons, detection
algorithms, and a summary of success rates.  Defer analysis of results
to next chapter.

