\section{Literature Review}
\label{sec:lit-review}

After -detailed search strategy-, I was unable to identify any studies which 
quantified indigenous Australian seasons.
I thus review a number of related fields, explaining the relevance of each and 
drawing together a summary of this `gap at the intersection'.

MUST have table here pointing to other parts of the literature review, 
explaining that concrete parts are in the related methods section to avoid 
misunderstanding. 

Possible sections – to be reorganised and mutated:
\begin{itemize}
\item Cross cultural and qualitative methods
\item Tropical seasonality, including monsoon and ENSO
\item Anticipated benefits and applications of the research
\item Relationship to phenology and historical climate change
\item Indigenous seasons literature (summary, also compare and contrast)
\item Overview of calendars and seasons – what are they and how are they defined?
\end{itemize}



The literature review surveys a range of fields relevant to indigenous 
calendars, which I deal with in three categories.  I begin by exploring the 
variety of ways in which calendars can be defined, before turning to 
seasonality and some of the ways seasonal cycles have been understood.  In the 
second part I look at Indigenous knowledge systems and then specifically 
seasonal calendars, which in Australia take a distinctively different approach 
to the European tradition.  Finally, I canvas the concrete applications of 
seasonality and the potential benefits of this research.



\subsection{Indigenous Knowledge}

There are a variety of terms used in the literature to describe Indigenous 
ecological knowledge systems:  \citet{clarke2009} refers to `land-based knowledge', 
\citet{petheram2010} and \citet{turner2009} use `traditional ecological 
knowledge', while \citet{cochran2015} simply use `indigenous knowledge'.  
\citet{berkes2012} defines Indigenous Ecological Knowledge (IEK) as ``a cumulative 
body of knowledge, practice and belief, evolving by adaptive processes and 
handed down through generations by cultural transmission''.  I prefer 
`Indigenous knowledge' to emphasise that ecology is not the only subject of 
Indigenous knowledge – for example, seasonal calendars often have cultural and 
economic as well as ecological importance.

There is a growing recognition among ecologists, natural resource managers, and 
scholars worldwide that Indigenous peoples hold important knowledge about the 
natural environment. The literature on IEK is well-established and vibrant.  In 
Australia and around the world, the value of Indigenous knowledge is recognised 
in areas as diverse as climate change and sustainability assessment
\citep[eg.][]{cochran2015}, holistic fire management \citep[eg.][]{clarke2009,price2012}, 
customary economic activities including aquaculture \citep{woodward2012a}, and 
natural resource management \citep[eg.][]{prober2011}.  The \textit{Environment 
Protection and Biodiversity Conservation Act 1999} suggests taking `a 
partnership approach to environmental protection and biodiversity conservation'
\citep{ens2012}, which is increasingly common at local and state levels.  

\citet{turner2009} distinguish Indigenous knowledge systems from `objective' 
scientific knowledge on the basis that Indigenous knowledge of practical 
matters is value-laden and observations or experience are tangled with beliefs, 
philosophy, law, and spirituality.  \citet{green2010a} and \citet{clarke2009} 
demonstrate the use of Indigenous phenological and climate knowledge to improve 
western scientific understanding of past and future climates.  

This literature provides a rich context for synthesis of indigenous knowledge 
and western climate science to investigate Australian seasonality.


\subsection{History, Seasons, and Phenology}
This section draws together a couple of ideas:  literature on historical 
climatology, the idea of what constitutes a season (weather, time, or ecology?),
and the role of phenology in determining seasons as a link in to Indigenous 
calendars.

Historical climatology is important in that it informs the methods I'm 
using, but also for the concept and to contextualise the research.  May lead to 
some comparisons between `lay knowledge' and IEK.\\

Seasons – specifically what makes something a season – is obviously key.  
Passage of time is one simplistic definition.  Others look at meteorology, or 
at ecology... which ties into phenology, which leads us back to Indigenous
Seasons in the next subsection!


I've started reading the literature on historical climatology, but don't have 
enough depth to write much yet.  It's going to be an important angle though. 

Matching oral history with numerical data is an established technique in 
climate science, as recorded observations can greatly assist in reconstruction 
of conditions before the instrumental record began.  In this case, observations 
within living memory are an excellent source, as the instrumental record is 
even today very sparse in Arnhem Land.  Investigation of social adaptation to 
local climate is an interesting and often closely linked field.

Look more into non-observational records; as far as I know dendrology etc has 
not been done much in the region; this is relevant as a missing alternative to 
oral histories.

Growing use of phenology in description of climate and climate change; similar 
use in IEK

Phenology is the study of timing in ecological events, usually associated with 
the first such event in a given year – such as the date of bloom for flowering 
plants, or the first sighting of migratory species.  Plant phenology is so 
closely associated with temperature (especially in temperate regions) or water 
availability (in arid regions) as to be a common proxy for measurement of such 
variables (REF).

Some phenological records in Europe go back centuries, and recent analyses show 
their significant value in understanding past climate conditions.
\citet{allstadt2015} look at leaf-out phenology as a definition of spring 
in the USA.  \citet{menzel2006} examine changes in butterfly presence in the
UK under observed climate change.

Indigenous knowledge of phenology stretches back millennia – and much is 
encoded in the seasonal calendars and traditions.


\subsection{Indigenous Seasons}
This section is not finished, but you get a pretty good idea where it's 
going.  Mainly it needs to be fleshed out with examples, references, etc. 

Indigenous seasonal calendars are qualitatively different to the European 
system, which was marked by lunar and later solar cycles.  By contrast, 
Australian indigenous seasonal calendars are defined by climatic and ecological 
cycles, with all the sensitivity to local context and variation between years 
that implies.

For practical use, a local and long-tested seasonal calendar is likely to have 
substantial advantages over one imported from Europe – at least for 
applications which depend on an understanding of Australia's environment, such 
as fire or natural resource management.

European seasons are adapted for agriculture in temperate regions.

Australian Indigenous calendars are generally adapted to a hunter-gatherer, 
more mobile lifestyle.  They also must cope with a far higher degree of 
interannual variation and offer more complex detail to be as useful.  It's no 
coincidence that the calendar is based directly on the important climate 
variables and marked by ecological events:  instead of ``the Wet is usually a 
good time for these foods'', the knowledge is more like ``after the first big 
storm (which marks this season), all the fish gather in these places''.



\subsection{Benefits of the Research: Seasons matter}
TODO - Still missing an elucidation of why the research matters to Yolngu; I'll 
get more of this picture in fieldwork (some for results, but some here too).
Generally needs more work in this section!\\

Seasons provide powerful heuristics for the expected range and variability in 
weather for a given time of year.

-	Implications of seasonality, relationship to NRM, applications, meaning of 
change, what does this way of thinking translate into action


Quantitative knowledge of local indigenous seasonality can improve practises in 
natural resource management, fire control, agriculture, and tourism – 
particularly outside the temperate zone.  Seasons can provide powerful 
heuristics for the expected range and variability in weather, and may yield 
insights into patterns of change.  For practical use, a local and long-tested 
seasonal calendar has substantial advantages over one imported from Europe.


See Barber 2005 ``where the clouds stand'' (about Tiwi Islands)

This research is expected to make a novel contribution to knowledge, raise 
awareness of Yolngu culture, and improve understanding of the climate 
seasonality of NE Arnhem Land.  The methods demonstrate two-ways research in 
combining indigenous knowledge with numerical data to characterise seasons.

For non-technical communities, both Indigenous and non, discussing climate 
change in terms of seasonality is more meaningful than degrees of warming.  
Framing the impacts of climate change in terms relatable to lived experience 
has the potential to boost engagement with and understanding of the major 
results of climate science – after all, everyone has some experience of the 
weather!

For the Yolngu community, increased engagement with the science around climate 
impacts has benefits for both adaptation choices and advocacy for mitigation, 
which are particularly important for coastal communities living off the land.  
Yolngu have expressed strong interest in other records of and perspectives on 
past environmental changes.  

For non-indigenous Australians, this research contributes to the continuing 
process of learning to live on this continent, discarding `maladaptive traditions'
- eg our poor infrastructue planning for drought and fire (REF the future eaters).


