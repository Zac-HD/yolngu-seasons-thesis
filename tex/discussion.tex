\chapter{Discussion}
\label{ch:discussion}

The discussion is divided into four sections.  First, interpretation and
significance of the results -- including some comment on experimental methods
which were not used in the final analysis.  Second, reflections on the methods
which \emph{were} used, and on the limitations of this research from both
scope and methodology.  Third, an attempt to assess potential applications or
benefits of this research.  Finally, several suggested directions for future
research in this area.



\section{Discussion of Results}
\label{sec:disc-results}

\subsection{Qualitative and Interview Results}

Interviews and informal discussions were an essential and integral part of
this study.

Two goals necessitated fieldwork and in-person discussion of Yolngu seasons.
The first goal of fieldwork  was to learn about the structure of Yolngu
seasons and try to catch any important information that would otherwise
fail to cross cultural barriers.  The volume of \cref{sec:complex-seasons},
on \textit{\nameref{sec:complex-seasons}}, speaks to the importance of these
results.
%
This validated the methodology -- without direct conversations, or if I had
assumed that published material set out all relevant information, crucial
information be missing and the results would only be of any use to the extent
that their accurate portrayal of Yolngu knowledge was irrelevant.
%
In short: there is little to no value in quantitative or numerical analyses
founded on faulty premises, no matter how rigorous or technically impressive.
An investment in cross-cultural understanding is well worth making, especially
on complex topics that appear to be simple!

The second goal was to collect detailed descriptions of the characteristics,
definitions and timing of each season, and investigate related issues
including extreme events and climate change.  The qualitative results
show this was a success.  However, this is in a sense preliminary work --
enough to get exploratory results, that can be taken back to Indigenous
research participants and partners to start a deeper discussion.  Future
studies may base such preliminary analysis entirely on secondary sources
such as \textit{Man of All Seasons} \citep{davis1989} and this thesis,
allowing researchers to demonstrate their intentions.


This study goes further and deeper than previous published descriptions of Yolngu seasons.
\citet{barber2005} and the \textit{Yan-nhangu Atlas} \citep{atlas2014} each
provide a list of seasons, with names and typical conditions on a single page.
\citet{davis1989} goes into much greater detail on the weather conditions and
social or ecological events of each season, but spends little time examining
the structure or fluidity of Yolngu seasons.  Davis is nonetheless unusually
detailed; some sources \citep[eg.][]{BOM-iwk} list seasons in terms of which
months they occur in, without further explanation of their characteristics,
onset definitions, or underlying variability .
\todo{add crisp statement of my unique or distinctive findings}


\subsection{Raw Weather Observations}
\label{ssec:disc-weather}

The observational weather record, shown in
\cref{fig:galiwinku-observations,fig:milingimbi-observations}, displays
several interesting patterns.  Each variable has a distinct seasonality,
the timing and in some cases intensity of which visibly varies between
years.  The dates of first and last rainfall vary by months; in some
years rain falls steadily and in others all at once.
%
Minimum and maximum temperatures follow slightly different patterns at
Galiwinku to those at Milingimbi -- for example, while both record their
highest temperatures around November, the following months are consistently
cooler at Galiwinku.

Wind in NE Arnhem Land appears to be dominated on a seasonal basis by the
monsoonal winds, and at these locations is heavily influenced by the sea
breeze.  Background conditions can be seen in 9am wind, which is minimally
affected by the sea breeze (see \cref{fig:galiwinku-seabreeze-direction}
for three-hourly data; speed and other stations in electronic appendices)
and displays a westerly or south-easterly regime depending on the monsoon.
Morning wind speed is steady throughout the year; at Galiwinku heavy rain
is associated with stronger winds.
%
At 3pm, things are more complicated.  The sea breeze blows from the north
for most of the year; but shows a south-easterly influence from May to
August.  Given that this is stronger at Galiwinku, which is around 80km
east of Milingimbi, I attribute this to a combination of the monsoon wind
and the sea breeze off the Gulf of Carpentaria, blowing across the intervening
land.  Speeds are similar throughout the day at Milingimbi, while Galiwinku
peaks in the afternoon.



\subsection{Detection of Seasons}
\label{subsec:disc-season-detection}

Without published literature as a guide, several approaches to numerical
detection of seasons were tested; conceptually based on (1) weather conditions
directly, (2) trend change in conditions, or (3) event- or threshold-based
detection.  Extensive exploratory analysis demonstrated that direct assessment
from daily weather was the most robust approach, and had the strongest
correspondence with qualitative and interview data.
%
Trend-based detection would in principle detect seasons similarly in years
which were eg. consistently hotter or colder than the norm, by recognising
a relative increase or decrease in the weather variable of interest.
In practice, this approach was highly dependent on arbitary parameters.
Most were too stable, without the possibility of `interrupting' seasons --
in some cases not changing at all.  Other parameters led to constantly
changing seasons, while still getting occasionally failing to change for a
year or more.  Event-based or trend detection is therefore not supported by
the qualitative definitions of Yolngu seasons.
%
Many of these problems could be disguised by adding a notion of time to the
season-detection algorithms.  However, this would impose a non-Indigenous
character, inconsistent with Yolngu participants explanations that the date
is fundamentally unimportant to Yolngu seasons.


The suite of algorithms developed to detect seasons (\cref{sec:applying-seasons-method})
appears to give good results, noting that an objective assessment is not
possible without some independent record of season occurence (which as
mentioned in \cref{meth:seas-detection} likely does not exist).
%
The season detection figures above show a relatively clear pattern for the
dry seasons, with the order and time-of-year of Midawarr, Dharrathamirri, and
Rarrandharr closely matching participant descriptions and published sources.
Dhuludur, the start of the wet season, is less clear, but this seems largely due to variation in timing between
years rather than poor detection.  If Barramirri and Mayaltha were treated as
a single season, they would be detected reasonably well -- but internally,
the distinction is very poor and their ordering is inconsistent. It is unclear
whether this is an arifact of the skewed distribution of rainfall,
or if it has a real basis in Yolngu definitions of seasons.


The tendency of this approach to show seasons interrupting each other at odd
times is a strength to exactly the degree that these interuptions accurately
reflect Yolngu understanding of the seasons.  Assessing this aspect of the
results would be a high priority for subsequent rounds of fieldwork, which
were not possible within the time and resources available for this thesis.

In the case that detected interruptions were too common, three solutions could be explored.
If phenological data were available, season definitions accounting for
ecological events might constrain the periods in which seasons can be detected.
This would be a fuller reflection of Yolngu knowledge, but faces serious
practical challenges.  In addition, if historically typical timings are used rather
than live data the results will fail to represent the `strange changes'
that participants reported in their environment.
%
The second  possibility is to apply a more complex algorithm to weather data,
which prefers to detect longer runs of days for each season.  The disdvantages
of this approach include increased complexity, and either heightened sensitivity
to initial conditions or a loss of locality -- in either case, the result for
a certain day could be affected by observations at a substantial temporal distance.
%
\todo{
Third, ...
real time testing / creating a validation dataset via informant data gathering
over one or more years.
}


\subsection{Quantitative Analysis}
\label{subsec:disc-season-characterisation}

\begin{figure}[p]
    \centerline{
    \includegraphics[width=1.1\linewidth]{galiwinku/timing-circle.pdf}}
    \caption[Three ways to characterise timing of seasons, Galiwinku]{
        Three ways to characterise timing of seasons, Galiwinku.
        The two rings in the center show the dominant season for each day,
        defined as that which was most often observed on that day (inner)
        and that with the greatest mean normalised index (outer).
        The arcs show intervals in which a specified proportion of days
        of that season fell -- thin line 90\%, medium 75\%, thick 50\% --
        which is the preferred approach to characterising season timing.
        The colour scheme is given by \cref{fig:galiwinku-season-pie}.

        \todo{rasterise inner rings to prevent further mis-printing}
        }
    \label{fig:galiwinku-timing-circle}
\end{figure}

The onset and timing of Indigneous seasons is typically presented (if at all)
by listing the months in which each season occurs.  Dissatisfaction with this
approach contributed significantly to the motivation for this research.
%
An early draft suggested calculating the onset date of each season in each
year, and summarising this data.  Unfortunately, such summaries are not
meaningful when seasons can interrupt one another in any order, implying
that identifying the first or last date on which a season occurred is not
a suitable approach.

How then, if at all, can the typical timing of Yolngu seasons be accurately
characterised?  The results in \cref{sec:applying-seasons-method} suggest two
basic possibilities: for each day of the year, take the mean over all years
and compare either the mean normalised indices, or the observed occurance
of each season.
%
Comparing the indices has a conceptual clarity to it; it suggests that the
typical timing of a season is that period where conditions tend to fit the
definition of that season better than any other.  It can be estimated simply
by reading off the topmost line in \cref{fig:season-daily-index} for any
particular day.  This aggregation gives a steady progression for most of the
year, and while it jitters a little in the Wet no two seasons ever co-occur.

Comparing observed occurance (see \cref{fig:season-daily-prob}) is somewhat
further from the raw data, but closer to people's experience.  If a similar
result to the index-based characterisation is desired -- non-overlapping
occurance, albeit with some interruption of seasons -- the same approach of taking the season
with the highest value for each day of the year can be applied.
%
Alternatively, for each season we can calculate the shortest interval which
contains some proportion of all days on which that season occured -- eg.
ninety percent for a robust and overlapping intervals, or fifty percent for
shorter intervale with gaps between them.  On balance, this last option
appears the most robust way to characterise typical timing of Yolngu seasons while
remaining well-founded in Indigenous definitions of the seasons.
%
All three approaches are drawn for comparison in \cref{fig:galiwinku-timing-circle}.



\section{Reflection -- methodology and limitations}
\label{sec:disc-reflection}

While the detail of specific methods is assessed above, reflections on the
methodology and scope are also important in exploratory research.  The
multi-stage approach was highly flexible, and this coupling between stages
made the study responsive to the unexpected qualitative results.  I hypothesise
that participant-led conversations and this flexibility were the primary
factors that led to this study's emergent focus on the complex structure of
Yolngu seasons -- an aspect of Indigenous seasonal knowledge which is neglected
by most of the available literature.

I consider that the aim of this research -- to develop an approach to
integrating traditional and quantitative/analytical knowledge -- was well
served by the chosen methodology and methods.  Multi-stage and iterative
methods are well suited to exploratory research, and can provide a robust
basis for further research.  Suggested improvements to the methods within
each stage, for investigation of Indigenous seasons specifically, are given
in the section above.  The scope of this thesis transitioned from broard
synthesis to focussed inquiry as the depth and complexity of Yolngu seasonal
knowledge became clear.  The final scope focusses on well-grounded analysis
of the data available, while identifying many tangents as potential directions
for future research.

The main limitations are that the results barely scratch
the depth of Indigenous Knowledge, it is difficult to assess the robustness
of the findings, and only limited generalisation is possible.  These
limitations imply interpretation should be cautious.  It is possible that
critical parts of Yolngu seasonal knowledge are missing from the results
above due to the small sample size and other limits.
%
Difficulty in assessing the results is more concerning, and derived from
uncertainty as to what the results \textit{should} be.   Properly resourcing
follow-up fieldwork should be a priority for any future study; a second round
of interviews would allow correction of any issues in the qualitative results.
It could also validate the thresholds chosen in the numerical definitions of
seasons, or offer feedback as they were changed.  Most importantly it would
ensure that the research is a partnership between equals by making follow-up
fieldwork an integral part of the research rather than a seperate undertaking.

The same detail and locality that embeds ecological and climate knowledge in
Indigenous calendars makes generalisation highly challenging.  Specific
findings of this study may not even be true for Yolngu calendars generally,
due to the focus on Galiwinku and Milingimbi.  More general findings, such as
the three-level typology of `kinds of seasons' may be more robust; for example
that pattern is recognisable in the Tiwi Seasons (\cref{fig:tiwi-seasons})
but would be surprising in temperate calendars.  Many of the premises, notably
including links between seasonal and ecological knowledge, are drawn from the
literature and appear to be reliable across a range of Indigenous groups.



\section{Directions for Further Study}
\label{sec:further-study}

Having demonstrated the viability of quantitative analysis of Indigenous
seasons, and discussed the value of such research, there are three distinct
directions for further study:  increasing one or more of the breadth, depth,
or extent of the research.  Alternative topics may also address the aim
of this research -- \textit{``developing an approach to integrating traditional
knowledge with a quantitative analytical approach''} -- but are not suggested
here.

\todo{depth, then breadth}

\subsection{Breadth of Study}
Increasing the breadth of study would result in similar but less exploratory
research; essentially trading novelty for rigor.  The clearest avenue for
this is to collect and analyse more data.

\begin{description}
\item[Multiple Indigenous calendars]
    Replication of this study with other Indigenous calendars would be
    valuable in each case, and the ability to compare calendars and investigate
    possible correlations between local climate and how seasons are defined
    could yield fascinating insights.

\item[Single calendar, multiple sites]
    Similarly, comparing the calendar of a single language group across
    multiple sites and subtly different climates could be very interesting.
    An `objective' weather record could really help researchers to draw out
    the differences between calendars, and perhaps see some of the subtleties.
    While this study had participants and weather observations from multiple
    sites, it did not analyse them as qualitatively distinct records or
    attempt any comparisons.

\item[Iterative fieldwork]
    Future work should if at all possible invest the time and resources
    neccessary for multiple visits to Indigenous homelands.  Second opinions,
    feedback on quantitative results, and the strong relationships that
    long-term engagement can grow could all prove surprisingly valuable.
\end{description}

\subsection{Depth of Study}
Deeper studies would likely focus on Indigenous Knowledge as it
relates to seasonal patterns, and work to integrate IK with western scientific
knowledge.  Researchers must ensure that both Indigenous and non-Indigenous
people benefit from such work, and not continue the history of exploitative
`sharing' condemed by eg. \citet{smith1999}.  Further analysis of monsoon-
or weather-based seasons would also fit in this category.

\begin{description}
\item[Attention to Yolngu benefits]
    \todo{good two-ways research would include working with Yolngu to determine
    and implement outcomes that are important to them.}

\item[Kinds of seasons]
    \Cref{sec:complex-seasons} describes a typology of seasons -- monsoonal,
    meteorological, and ecological -- that can be identified as part of a
    `single' Yolngu calendar (to the degree that `single Yolngu calendar'
    is a coherent idea).  Further study on the conceptual basis of this typology would
    be enlightening, as would gaining a stronger understanding of how
    seasonal indicators interact.

\item[Ecological knowledge]
    Investigating Indigenous ecological seasons could provide deep insights
    into local ecology and traditional natural resource management. The
    principal challenges are logistical:  such study would require real
    investment and relationship-building, and the collecting data
    required to `translate' Indigenous Knowledge to western science would
    be expensive.

\item[Per-year patterns]
    There is clearly a degree of variation in the ways seasons occur between
    different years, study of which may reveal patterns.  Are there distinct
    regimes that describe seasonal onset, timing, or interruption?  If so, what
    data is required to categorise a year by regime and make useful predictions?
\end{description}

\subsection{Extent of Study}
Increasing the extent of study would mean expanding the topic; perhaps by
looking at links to the wider context -- climate indices or long-term
change, non-Indigenous calendars, and so on.

\begin{description}
\item[Non-Indigenous calendars]
    A comparative study of many kinds of calendars could be fascinating --
    Indigenous, agricultural, solar, lunar, or harvest seasons -- unlocking
    the knowledge embedded in each.

\item[Climatology]
    With seasons defined by weather observations, a climatological study of the
    Yolngu calendar could suggest links between forecastable climate indices
    and Arnhem Land ecology.  Research could begin by investigating correlations
    between seasonal occurance and non-surface weather, or look at cycles with
    timescales on the order of months (eg. the Madden-Julian Oscillation), ENSO,
    the Indian Ocean Dipole, or even the Pacific Decadal Oscillation.

\item[Climate Change]
    The methods used in this study can be applied to any dataset which
    includes a daily record of the relevant variables.  This enables
    investigation of observed, estimated, or projected past and future changes
    across scenarios including a wider range of conditions than exist in the historical record.
    One obvious application of this is to investigate how Yolngu seasons have
    changed over the past century, and how they may change under future climate
    change scenarios in model forecast and backcasts.  Yolngu participants
    indicated substantial interest in such studies.
\end{description}



\section{Benefits and Applications of the Research}
\label{sec:applications-benefits}

\todo{re-arrange to have benefits, then applications? (S)
Ensure benefits for Yolngu are clearly stated}

This thesis makes a novel contribution to academic knowledge: it is the
first study to quantitatively assess Australian Indigenous seasons, and
develops techniques for further novel analysis including characterisation
of seasonal timing.
%
It supports certain concrete applications of interest to both Yolngu and
non-Indigenous people, particularly through the methods pioneered above
rather than the specific results.  Intangible benefits are more difficult
to identify, but likely to be important.


Considering what actitivites in northern Australia might benefit from the
insights of this research reveals several potential applications.  This strategy is
founded on the idea that a local and long-tested seasonal calendar has real
advantages for land, environmental, and natural resource management compared
to an imported temperate-zone calendar assuming low inter-annual variability.
%
With the exploratory approach taken by this study, and the context-specific
nature of Indigenous seasons, it is anticipated that most concrete application
would involve stakeholder groups conducting their own research based on the
methods developed in this thesis rather than reading specific results.


Four industries are identified which might benefit substantially from follow-on
research, using seasons as powerful heuristics for the expected range
and variability in weather, or to provide insights into patterns of change.
%
Land or natural-resource management is deeply affected by seasons; while
specific applications are difficult to predict there are clear advantages
to understanding and predicting local conditions.
%
Second, these methods for integrating Indigenous Knowledge with numerical --
and forecastable -- weather data have clear applications in fire management.
In joint-management areas, internal research could produce predictive tools
for fire risk or to recomend proscribed burning, based on both Indigenous
Knowledge and cutting-edge fire science.

Northern agriculture, especially for native or other non-traditional crops,
could benefit from a clear understanding of local seasonality.  Research in
this field should pay close attention to the relationships between vegetation
dynamics and weather, and might improve the profitability and viability of Indigenous
enterprises.
%
Finally, Indigenous Knowledge and culture drives a substantial part of the
tourism industry in Arnhem Land and across northern Australia, with some
operators already employing ecologists and Indigneous guides.  The depth and
richness of Indigenous Knowledge may be better appreciated by visitors if
it is presented alongside and integrated with scientific knowledge, which
may be more widely recognised or respected.


The applications and benefits of this research for Yolngu people also deserve
consideration.  Participants spoke about the need for reciprocity in research
and expressed strong interest in scientific records of and perspectives on
environmental change -- and their desire for non-Indigenous Australians to
learn about Indigenous ways of living, just as they have had to learn a to
live in a non-Indigenous system.
\begin{quote}
    I would like to see benefit for both of us from this project, because I am
    sharing and my help should be acknowledged.  You need to share what you learn. ...
    [The] wet season and the dry season changes, it's getting harder to read
    because there's more foreign stuff in Australia.  But we have to learn!
    We have to try catching up to the changes, generation to generation.
\end{quote}
Reflection on \citet{petheram2010} suggests this approach to seasonal knowledge
may assist communities to engage with climate change scicence, to make
adaptation choices and as a platform to advocate mitigation policy -- both
important for coastal communities living off the land.  There may be other
ways for Yolngu to apply this research, the specifics of which are a matter for
community discussion and further research.


Beyond specific applications, this research has benefits.  It
demonstrates more nuanced ways of thinking about tropical seasons than the
simplistic Wet/Dry dichotomy \citep{willmett2009}, and builds the foundation
for new framing around climate change -- everyone has lived experience with
weather and seasons, where global average degrees of warming is merely jargon.
%
In the long term, it may contribute to the long and difficult process of
non-Indigenous Australians learning to understand, live in, and nurture
Australia through a fully Australian world-view.  This would require setting
aside our `maladaptive traditions' grounded in non-Australian contexts
\citep{flannery1994} to celebrate the unique challenges presented by our variable and extreme
environments.  Imagine a future where local radio discusses not the timing of
European flowers, but seasons -- with a guest panel of Indigenous elders and
climate scientists.
%
If this research can inspire the building of bridges or the breaking down of walls
between our many seperate forms of knowledge, Australia may be better placed
to meet the challenges of the coming century than the last.

