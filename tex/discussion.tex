\chapter{Discussion}
\label{ch:discussion}

The key findings of this research are:
\begin{itemize}
\item Yolngu seasonal calendars are complex.  They have three kinds of
    seasons, each with their own indicators and typical timescale.  Seasons
    occur differently from year to year.  Calendars vary between locations.

\item A Yolngu calendar defined by quantitative weather observations can be constructed,
    and is consistent with interview data.  Novel analysis demonstrates that
    these definitions can be applied to observational weather data to derive
    a record of season occurance.

\item Consistent with qualitative data, the seasons record is complex and
    shows seasons interrupting each other without a clear order.  Multiple
    approaches for characterising `typical' seasonality are explored, and
    two are considered well-grounded.
\end{itemize}

In \cref{sec:disc-results}, these findings are interpreted and discussed in
the context of relevant literature.  \cref{sec:disc-reflection} reflects on
the methods and methodology developed in this thesis, and the limitations
of the research.  Benefits and potential applications of the research are
discussed in \cref{sec:applications-benefits}.


\section{Context and Interpretation of Findings}
\label{sec:disc-results}

\subsection{Qualitative and Interview Results}

Interviews and informal discussions were an essential and integral part of
this study.

Two goals necessitated fieldwork and in-person discussion of Yolngu seasons.
The first goal of fieldwork  was to learn about the structure of Yolngu
seasons and obtain any important information that might otherwise
fail to cross cultural barriers.  The length of \cref{sec:complex-seasons},
on \textit{\nameref{sec:complex-seasons}}, speaks to the importance of these
results and validates the research approach.
%
Without direct conversations, or if I had assumed that all relevant information
could be gathered from published material, crucial information would have been
missed.  This would seriously impair the results, as there is little to no value
in quantitative or numerical analyses
founded on faulty premises, no matter how rigorous or technically sophisticated.
An investment in cross-cultural understanding is well worth making, especially
on complex topics that appear to be simple on superficial investigation.

The second goal was to collect detailed descriptions of the characteristics,
definitions and timing of each season, and investigate related issues
including extreme events and climate change.  The qualitative results
show that this was a success.  However, in a sense this is preliminary work --
enough to obtain exploratory results, that can be taken back to Indigenous
research participants and partners to inform a deeper second-stage discussion.  Future
two-ways research might base such preliminary analysis entirely on secondary sources
such as \textit{Man of All Seasons} \citep{davis1989} and this thesis,
allowing researchers to demonstrate their intentions.


This thesis goes further and deeper than previous published descriptions of Yolngu seasons.
\citet{barber2005} and the \textit{Yan-nhangu Atlas} \citep{atlas2014} each
provide a list of seasons, with names and typical conditions on a single page.
\citet{davis1989} goes into much greater detail on the weather conditions and
social or ecological events of each season, but spends little time examining
the structure or fluidity of Yolngu seasons.  Davis is nonetheless unusually
detailed; some sources \citep[eg.][]{BOM-iwk} list seasons by months of
occurance, without further explanation of their characteristics,
onset definitions, or underlying variability .


This research has created the first known record of season occurance by day,
based on season definitions and observational weather data.  This enables
investigation of the short-term `interruptions' of seasons, and precise
characterisation of both typical timing and variation in timing.  Without
such data, no previous study \citep[eg.][]{davis1989,barber2005,CSIROcals,BOM-iwk}
has been able to characterise Australian Indigenous seasons in this way.



\subsection{Weather Observations}
\label{ssec:disc-weather}

The observational weather record in the study area, shown in
\cref{fig:galiwinku-observations,fig:milingimbi-observations}, displays
several interesting patterns.  Each variable has a distinct seasonality,
the timing and in some cases intensity of which varies visibly between years.
The rainy period begins at substantially different times from year to year,
and rainfall intensity also appears to vary.  This is consistent with known
interannual variability of the monsoon, eg. \citet{cook2001}.
%
Minimum and maximum temperatures follow slightly different patterns at
Galiwinku to those at Milingimbi -- for example, while both record their
highest temperatures around November, while the following months are consistently
cooler at Galiwinku.

Wind in NE Arnhem Land appears to be dominated on a seasonal basis by the
monsoon, and is heavily influenced by the sea breeze at these
locations.  Background conditions can be seen in 9am wind, which is minimally
affected by the sea breeze and displays a westerly or south-easterly regime
depending on the phase of the monsoon.  See \cref{fig:galiwinku-seabreeze-direction}
for three-hourly data; electronic appendices for wind speeds and data for other stations.
Morning wind speed is steady throughout the year; at Galiwinku heavy rain
is associated with stronger winds.
%
At 3pm the situation is more complicated.  The sea breeze blows from the north
for most of the year, but shows a south-easterly influence from May to
August.  This is consistent with the seasonal northwards migration of the ITCZ,
and resulting dominance of the south-east trade winds over the area in these
months.  Speeds are similar throughout the day at Milingimbi, while at Galiwinku
wind speed peaks in the afternoon.


This analysis builds on weather-based characterisations of north Australian
seasonality by \citet{cook2001} and \citet{holland1985}, considering a wider
range of surface-level indicator variables than previous studies which were
generally confined to rainfall or high-altitude wind.  Notably, this thesis
is grounded in Yolngu knowledge based on generations of observation rather than
descriptive statistics which may not capture the fine variability of seasonal
characteristics.



\subsection{Detection of Seasons}
\label{subsec:disc-season-detection}

In the absence of published literature as a guide several approaches to numerical
detection of seasons were tested, conceptually based on (1) observed weather conditions
at a specific time, (2) trend change in conditions, or (3) event- or threshold-based
detection.  Extensive exploratory analysis demonstrated that direct assessment
from daily weather was the most robust approach, and was well grounded in the
qualitative and interview data.
%
Trend-based detection would in principle detect seasons similarly in years
which were, for example, consistently hotter or colder than the norm by recognising
a relative increase or decrease in the weather variable of interest.
In practice, this approach was highly dependent on arbitary parameters.
Most were too stable, excluding the possibility of `interrupting' seasons --
in some cases not changing at all.  Other parameters led to constantly
changing seasons, while still occasionally failing to change for a
year or more of the observational record.  Event-based or trend detection
is therefore not supported by the qualitative definitions of Yolngu seasons.
%
Many of these problems could be disguised by adding a notion of time to the
season-detection algorithms.  However, this would impose a non-Indigenous
character on the analysis, inconsistent with Yolngu explanations that the date
is fundamentally unimportant to Yolngu seasons.


The suite of algorithms developed to detect seasons (\cref{sec:applying-seasons-method})
appears to give good results, noting that an objective assessment is not
possible without some independent record of season occurence (which as
mentioned in \cref{meth:seas-detection} likely does not exist).
%
The season detection analyses
(\cref{fig:galiwinku-seasons,fig:season-daily-index,fig:season-daily-prob,fig:galiwinku-timing-circle})
show a relatively clear pattern for the
dry seasons, with the order and time-of-year of Midawarr, Dharrathamirri, and
Rarrandharr closely matching participant descriptions and published sources.
Dhuludur, the start of the wet season, is less clear, but this seems largely due to variation in timing between
years rather than poor detection.  If Barramirri and Mayaltha were treated as
a single season, they would be detected reasonably well -- but internally,
the distinction is poor and their ordering is inconsistent. It is unclear
whether this is an artifact of the skewed distribution of rainfall,
or if it has a real basis in Yolngu definitions and perceptions of seasons.


Applying this approach tends to show seasons interrupting each other, which may
seem strange to non-Indigenous analysts.  To the degree that this accurately
reflects Yolngu understanding of the seasons, it is a strength -- but this is
difficult to assess in the absence of validation data.  Testing this aspect of the
results would be a high priority for subsequent rounds of fieldwork, which
were not possible within the time and resources available for this research.

If interruptions were detected too frequently, three solutions could be explored.
Phenological data could provide a basis for ecological season definitions,
which might constrain the periods in which seasons can be detected.
This would be a fuller reflection of Yolngu knowledge, but faces serious
practical challenges including that if historically typical timings are used
(rather than live data) the results will fail to represent the `strange changes'
that participants reported in their environment.
%
The second possibility is to apply a more complex algorithm to weather data,
which preferentially detects longer runs of days for each season.  The disdvantages
of this approach include increased complexity, and either heightened sensitivity
to initial conditions or a loss of locality -- in either case, the result for
a certain day could be affected by observations at a substantial temporal distance.
%
Finally, a validation dataset could be created over one or more years of
additional fieldwork or surveys, including data collection over SMS or web
platforms.  This dataset would be valuable in strengthening two-way learning,
refining the methods, and supporting further study including the options above.
Collecting participant observations on ecology as well as meterological
season would be particularly useful.


\subsection{Quantitative Analysis}
\label{subsec:disc-season-characterisation}

\begin{figure}[p]
    \centerline{
    \includegraphics[width=1.1\linewidth]{galiwinku/timing-circle.pdf}}
    \caption[Three ways to characterise timing of seasons, Galiwinku]{
        Three ways to characterise timing of seasons, Galiwinku.
        The two rings in the center show the dominant season for each day,
        defined as that which was most often observed on that day (inner)
        and that with the greatest mean normalised index (outer).
        The arcs show intervals in which a specified proportion of days
        of that season fell -- thin line 90\%, medium 75\%, thick 50\% --
        which is the preferred approach to characterising season timing.
        The colour scheme is given by \cref{fig:galiwinku-season-pie}.
        }
    \label{fig:galiwinku-timing-circle}
\end{figure}

In the best published sources, onset and timing of Indigneous seasons is typically presented
by listing the months in which each season occurs.  Dissatisfaction with this
approach contributed significantly to the motivation for this research.
%
While calculating seasonal onset dates each year and summarising these data
is possible, such summaries are not meaningful when seasons can interrupt
one another in any order.  Identifying the first or last date on which a
season occurred is not a suitable approach given the detail and complexity of Yolngu seasons.

How then, if at all, can the typical timing of Yolngu seasons be accurately
characterised?  The results in \cref{sec:applying-seasons-method} suggest two
basic possibilities: for each day of the year a mean over all years can be
compared to either the mean normalised indices, or the observed occurance
of each season.
%
Comparing the indices has conceptual clarity; it suggests that the
typical timing of a season is that period where conditions tend to fit the
definition of that season better than any other.  It can be estimated simply
by reading off the topmost line in \cref{fig:season-daily-index} for any
particular day.  This aggregation gives a steady seasonal progression for most of the
year, and while it `jitters' a little in the Wet no two seasons ever co-occur.

Comparing modelled occurance rates of Yolngu seasons (see \cref{fig:season-daily-prob}) is somewhat
further from the weather data, but closer to people's experience.  If a similar
result to the index-based characterisation is desired (i.e. non-overlapping occurance
with interruptions) the same approach of taking the season
with the highest value for each day of the year can be applied.
%
Alternatively, for each season the shortest interval which contains some
proportion of all days on which that season occured can be calculated --
for example ninety percent for robust and overlapping intervals, or fifty percent for
shorter but non-overlapping intervals.  On balance, this last option
appears the most robust way to characterise typical timing of Yolngu seasons while
acknowledging inter-annual variability and remaining well-founded in
Indigenous definitions of the seasons.

All three approaches are illustrated in \cref{fig:galiwinku-timing-circle}.
The two rings in the center show the dominant season for each day, defined as
that which was most often observed on that day (inner) and that with the
greatest mean normalised index (outer).  The arcs show intervals in which a
specified proportion of days of that season fell, which is the preferred
approach to characterising season timing.  \todo{highlight meaning of this}

\defcitealias{BOM-iwk}{BOM 2016}

The value and novelty of this analysis can be understood by contrasting it with the CSIRO
posters, for example the Tiwi Seasons Calendar (\cref{fig:tiwi-seasons}) on
page~\pageref{fig:tiwi-seasons}.  The timing of these seasons is represented
without showing variability or the potential for interruptions (which may not
be a part of this calendar), where \cref{fig:galiwinku-timing-circle}
illustrates both.
%
It is important to note that the CSIRO posters tend to focus on the ecological knowledge and
customary actvities associated with seasons (as do other published sources,
including eg. \citealp{barber2005}, \citetalias{BOM-iwk}, \citealp{davis1989}) rather than weather
characteristics and timing.  Just as this thesis covers substantially less
ecological knowledge than previous studies, it shows substantially more
about the timing and variability of Yolngu seasons than any other source --
different information, for a different purpose.


\section{Reflection -- methodology and limitations}
\label{sec:disc-reflection}

While the detail of specific methods is assessed above, reflections on the
methodology and scope are also important in exploratory research.

The flexibility of the multi-stage approach and coupling between stages
allowed the study to be adapted to unexpected qualitative results that were
identified during the interviews.  I consider
that participant-led conversations and this flexibility were the primary
factors that led to the emergent focus on the complex structure of
Yolngu seasons, an aspect of Indigenous seasonal knowledge that is neglected
in most of the available literature.  A close reading of \citet[][eg. p89]{barber2005}
does suggest a multi-level seasonal cycle between monsoonal Wet/Dry and
```short Yolngu seasons'', but this is not explored in depth.

I consider that the aim of this research -- to develop an approach to
integrating traditional and quantitative/analytical knowledge -- was well
served by the chosen methodology and methods.  Multi-stage and iterative
methods are well suited to exploratory research, and would provide a robust
basis for further research.  Suggested improvements to the methods within
each stage, for investigation of Indigenous seasons specifically, are given
in the section above.  The scope of this thesis transitioned from broad
synthesis to focussed inquiry as the depth and complexity of Yolngu seasonal
knowledge became clear.  The final scope focusses on well-grounded analysis
of the data available, while identifying many tangents and extensions as
potential directions for future research.

The main limitations are that the results represent a valuable but tiny fraction
of relevant Indigenous Knowledge, it is difficult to assess the robustness
of the findings, and only limited generalisation is possible.  These
limitations imply interpretation should be cautious.  It is possible that
critical parts of Yolngu seasonal knowledge are missing from the results
above due to the small sample size and other limits.
%
Difficulty in assessing the results is more concerning, and derived from
uncertainty as to what the results \textit{should} be.   Properly resourcing
follow-up fieldwork should be a priority for any future study
\citep[see eg.][on engagemnt plans]{jackson2015}.  A second round
of interviews or other appropriate data collection would allow engagement
with and correction of any shortcomings in the qualitative results.
It could also validate or guide refinement of the thresholds chosen in the
numerical definitions of seasons.  Most importantly it would
ensure that the research is a partnership between equals by making follow-up
fieldwork an integral part of the research rather than a seperate undertaking.

The same detail and locality that embeds ecological and climate knowledge in
Indigenous calendars makes generalisation highly challenging
\citep[][came to similar conclusions]{barber2005,davis1989}.  Specific
findings of this study may not even be true for Yolngu calendars generally,
due to the focus on Galiwinku and Milingimbi.  More general findings, such as
the three-level typology of `kinds of seasons' may be more robust; for example
that pattern is recognisable in the Tiwi Seasons (\cref{fig:tiwi-seasons})
but would not be expected in temperate latitude calendars.  Many of the general premises, notably
including links between seasonal and ecological knowledge, are drawn from the
literature and appear to be reliable across a range of Indigenous groups.




\section{Applications and Benefits of the Research}
\label{sec:applications-benefits}

This thesis makes a novel contribution to academic knowledge: it is the
first study to quantitatively assess Australian Indigenous seasons, and
develops techniques for further novel analysis including characterisation
of seasonal timing.
%
It supports concrete applications of interest to both Yolngu and
non-Indigenous people, particularly through the methods pioneered above
rather than the specific results of the calendar.  Intangible benefits
are more difficult to identify, but no less important.


Considering what activities in Northern Australia might benefit from the
insights of this research reveals several potential applications
\citep[see eg.][]{whitepaper}.  A local, experience-based, and tested seasonal calendar has real
advantages for land, environmental, and natural resource management compared
to an imported temperate-zone calendar assuming low spatial and inter-annual variability.
%
With the exploratory approach taken in this study, and the context-specific
nature of Indigenous seasons, it is anticipated that the most concrete applications
would involve stakeholder groups conducting their own research based on the
methods developed in this thesis rather than applying specific results of this research.


Four industries are identified which might benefit substantially from follow-on
research, using seasons as powerful heuristics for the expected range
and variability in weather, or to provide insights into patterns of change.
%
Land or natural resource management is deeply affected by seasons; while
specific applications are difficult to predict there are clear advantages
to understanding and predicting local conditions.
%
Second, these methods for integrating Indigenous Knowledge with numerical --
and forecastable -- weather data have clear applications in fire management
where seasonality is a key determinant of fire impacts \citep{driscoll2010}.
In joint-management areas, internal research could produce predictive tools
for fire risk or to recommend prescribed burning, based on both Indigenous
Knowledge and cutting-edge fire science \citep[see eg.][]{bowman2003,yibarbuk2001}.

Northern agriculture, especially for native or other non-traditional crops,
could benefit from a clear understanding of local seasonality.  Research in
this field should pay close attention to the relationships between vegetation
dynamics and weather, and might improve the profitability and viability of Indigenous
enterprises.  This would extend recent work \citep[eg.][]{jackson2012,jackson2015}
with Indigenous participating in water and river system management plans.
%
Finally, Indigenous Knowledge and culture drives a substantial part of the
tourism industry in Arnhem Land and across northern Australia, with some
operators already employing Indigneous ecologists guides.  The depth and
richness of Indigenous Knowledge may be better appreciated by non-Indigenous visitors if
it is presented alongside and integrated with scientific knowledge, which
may be more widely recognised or respected.  This integration could also
encourage the application of Indigenous knowledge to improve management outcomes.


The applications and benefits of this research for Yolngu people also deserve
consideration.  Participants spoke about the need for reciprocity in research
and expressed strong interest in scientific records of and perspectives on
environmental change -- and their desire for non-Indigenous Australians to
learn about Indigenous ways of living, just as they have had to learn to
live in a non-Indigenous system.
\begin{quote}
    I would like to see benefit for both of us from this project, because I am
    sharing and my help should be acknowledged.  You need to share what you learn. ...
    [The] wet season and the dry season changes, it's getting harder to read
    because there's more foreign stuff in Australia.  But we have to learn!
    We have to try catching up to the changes, generation to generation.
\end{quote}
Reflection on \citet{petheram2010} suggests the approach to seasonal knowledge in this thesis
may assist communities to engage with climate change science, to make
adaptation choices and as a platform to advocate mitigation of climate change impacts -- both
important for coastal communities living off the land.  There may be other
ways for Yolngu to apply this research, the specifics of which are a matter for
community discussion and further research.


Further research is recommended on Indigenous seasonal calendars, or
integrating Indigenous and non-Indigenous knowledge about other topics.  Such
research would be applicable to many of the challenges facing
Australia in the twenty-first century, helping Indigenous and non-Indigenous
people to understand and adapt to global change, and perhaps for non-Indigenous
Australians to reach a genuinely Australian understanding of what it means to
live on this continent, our home.

