\chapter{Discussion}
\label{ch:discussion}

\todo{write pretty much the whole chapter}

\section{Discussion of Results}
\label{sec:disc-results}

\subsection{Qualitative and Interview Results}

\todo{detailed interpretation of the qualitative results.
Focus on the unexpected - complex seasons.  This reinforces the
methodological importance of qualitative methods -- there's no
point at all to numerical analysis on wrong premises!
}


\subsection{Detection of Seasons}

\todo{
Note that trend- and event-based approaches were tested and didn't work.
}


\subsection{Quantitative Analysis}

\todo{discuss this stuff}



\section{Discussion of Exploratory Methods}
\label{sec:disc-methods}

\todo{What were the methods?  How well did they work?
Are they applicable to other calendars?  If so, to what extent?
}



\section{Relevance and Novelty}
\label{sec:disc-rel-novelty}

\todo{With results in hand, how much of the potential was fulfilled?}



\section{Limitations of the Research}
\label{sec:disc-limitations}

\todo{Write in general terms about what -- of the in-scope things --
were limitations or didn't work.  Where did scoping hurt?
What could or should have been done better?  How much was foreseeable?
Detailed critique belongs in specific sections above, though.
}



\section{Applications or Benefits of the Research}
\label{sec:applications-benefits}
\todo{Still missing an elucidation of why the research matters to Yolngu -
and everyone else!}
Looking at abstract benefits (contribution to knowledge, valuing Indigenous knowledge),
but also cover concrete benefits - what activities in northern australia
in particular (eg land and NR management) will benefit from these insights?
Key aspect: how transferable are these insights?

~\\

Seasons provide powerful heuristics for the expected range and variability in
weather for a given time of year.

-	Implications of seasonality, relationship to NRM, applications, meaning of
change, what does this way of thinking translate into action


Quantitative knowledge of local indigenous seasonality can improve practises in
natural resource management, fire control, agriculture, and tourism –
particularly outside the temperate zone.  Seasons can provide powerful
heuristics for the expected range and variability in weather, and may yield
insights into patterns of change.  For practical use, a local and long-tested
seasonal calendar has substantial advantages over one imported from Europe.


See Barber 2005 ``where the clouds stand'' (about Tiwi Islands)

This research is expected to make a novel contribution to knowledge, raise
awareness of Yolngu culture, and improve understanding of the climate
seasonality of NE Arnhem Land.  The methods demonstrate two-ways research in
combining indigenous knowledge with numerical data to characterise seasons.

For non-technical communities, both Indigenous and non, discussing climate
change in terms of seasonality is more meaningful than degrees of warming.
Framing the impacts of climate change in terms relatable to lived experience
has the potential to boost engagement with and understanding of the major
results of climate science – after all, everyone has some experience of the
weather!

For the Yolngu community, increased engagement with the science around climate
impacts has benefits for both adaptation choices and advocacy for mitigation,
which are particularly important for coastal communities living off the land.
Yolngu have expressed strong interest in other records of and perspectives on
past environmental changes.

For non-indigenous Australians, this research contributes to the continuing
process of learning to live on this continent, discarding `maladaptive traditions'
- eg our poor infrastructue planning for drought and fire (REF the future eaters).



\section{Directions for Further Study}
\label{sec:further-study}
\todo{intro paragraph, expand section and individual items}

\begin{description}
\item[Ecological seasons]
        Investigating Indigenous ecological seasons could provide deep insights
        into local ecology and traditional natural resource management.
        The principle challenges are resourcing, to collect the quantitative
        data required to `translate' back to western science, and lengthy
        investment in time and relationships required.

\item [More calendars]
        Replication of this study with other Indigenous calendars would be
        valuable in each case, and the ability to compare calendars and investigate
        possible correlations between local climate and how seasons are
        defined could yield fascinating insights.

\item [Single calendar, more sites]
        Similarly, comparing the calendar of a single language group across
        multiple sites and subtly different climates could be very interesting.
        An `objective' weather record could really help researchers to draw out
        the differences between calendars, and perhaps see some of the subtleties.

\item [Other meteorological datasets]
        The methods used in this study can be applied to any dataset which
        includes a daily record of the relevant variables.  This enables
        investigation of changes over time -- past or future -- and scenarios
        including a wider range of conditions than exist in the historical record.
        I plan to conduct and publish such a study using forecast and backcast
        model outputs from the ACCESS and CMIP5 ensembles immediately
        after graduating.

\item []


\end{description}



