\chapter{Discussion}
\label{ch:discussion}

The discussion is divided into four sections.  First, interpretation and
significance of the results -- including some comment on experimental methods
which were not used in the final analysis.  Second, reflections on the methods
which \emph{were} used, and on the limitations of this research from both
scope and methodology.  Third, an attempt to assess potential applications or
benefits of this research.  Finally, several suggested directions for future
research in this area.



\section{Discussion of Results}
\label{sec:disc-results}

\subsection{Qualitative and Interview Results}

Interviews and informal discussions are an essential part of this study --
that much was obvious in advance, and easily justified the cost of fieldwork.
In retrospect they were indeed essential, for unforseen reasons!
%
Two goals necessitated fieldwork and in-person discussion of Yolngu seasons.
The first was to collect detailed descriptions of each season, the definitions
and timing, extreme events, climate change, and so on; the qualitative results
show this was a success.  However, this is in a sense preliminary work --
enough to get exploratory results, that can be taken back to Indigenous
research participants and partners to start a deeper discussion.  In future
studies secondary sources such as \textit{Man of All Seasons} \citep{davis1989}
may support this preliminary analysis, and allow researchers to demonstrate
their intentions, at some risk of wasted effort.

The second goal of fieldwork  was to learn about the structure of Yolngu
seasons and try to catch any important information that would otherwise
fail to cross cultural barriers.  The volume of \cref{sec:complex-seasons},
on \textit{\nameref{sec:complex-seasons}}, speaks to the importance of these
results.
%
This validated the methodology -- without direct conversations, or if I had
assumed that published material set out all relevant information, crucial
information be missing and the results would only be of any use to the extent
that their accurate portrayal of Yolngu knowledge was irrelevant.
%
In short: there is little to no value in quantitative or numerical analyses
founded on faulty premises, no matter how rigorous or technically impressive.
An investment in cross-cultural understanding is well worth making, especially
on complex topics that appear to be simple!


\todo{Compare to previous research}



\subsection{Weather Observations}

The observational weather record, shown in
\cref{fig:galiwinku-observations,fig:milingimbi-observations}, displays
several interesting patterns.  Each variable has a distinct seasonality,
the timing and in some cases intensity of which visibly varies between
years.  The dates of first and last rainfall vary by months; in some
years rain falls steadily and in others all at once.
%
Minimum and maximum temperatures follow slightly different patterns at
Galiwinku to those at Milingimbi -- for example, while both record their
highest temperatures around November, the following months are consistently
cooler at Galiwinku.

Wind in NE Arnhem Land appears to be dominated on a seasonal basis by the
monsoonal winds, and at these locations is hevily influenced by the sea
breeze.  Background conditions can be seen in 9am wind, which is minimally
affected by the sea breeze (see \cref{fig:galiwinku-seabreeze-direction}
for three-hourly data; speed and other stations in electronic appendices)
and displays a westerly or south-easterly regime depending on the monsoon.
Morning wind speed is steady throughout the year; at Galiwinku heavy rain
is associated with stronger winds.
%
At 3pm, things are more complicated.  The sea breeze blows from the north
for most of the year; but shows a south-easterly influence from May to
August.  Given that this is stronger at Galiwinku, which is around 80km
east of Milingimbi, I attribute this to a combination of the monsoon wind
and the sea breeze off the Gulf of Carpentaria, blowing across the intervening
land.  Speeds are similar throughout the day at Milingimbi, while Galiwinku
peaks in the afternoon.



\subsection{Detection of Seasons}
\label{subsec:disc-season-detection}

Without published literature as a guide, several approaches to numerical
detection of seasons were tested; conceptually based on (1) weather conditions
directly, (2) trend change in conditions, or (3) event- or threshold-based
detection.  It eventually became clear that direct assessment from daily
weather was the most robust approach, as well as enjoying a stronger basis
in the qualitative and interview data.
%
Trend-based detection would in principle detect seasons similarly in years
which were eg. consistently hotter or colder than the norm, by recognising
a relative increase or decrease in the weather variable of interest.  In
practise, this approach was prone to very frequent `jitter' between seasons,
or with other parameters too stable to reflect the possibility of `interrupting'
seasons -- or sometimes that the season could change at all.
%
Event-based detection is not supported by the qualitative definitions of Yolngu
seasons, and has similar problems with `stickiness' -- some years would
not pass the event threshold, and seasonal change would not be detected!
%
Many of these problems could be disguised by adding a notion of time to the
season-detection algorithms -- unacceptable, on the basis that research
participants explained the fundamental unimportance of date to Yolngu seasons.


The algorithm which was ultimately used to detect seasons,
described in \cref{sec:applying-seasons-method}, appears to give good results
-- an objective assessment is not possible without some independent record of
season occurence, which as mentioned in the Methods chapter likely does not exist.
%
The season detection figures above show a relatively clear pattern for the
dry seasons, with Midawarr, Dharrathamirri, and Rarrandharr in the correct
dominant order and time of year.  Dhuludur, the start of the wet season, is
somewhat less clear -- but this seems mostly due to variation in timing between
years rather than poor detection.  If Barramirri and Mayaltha were treated as
a single season, they would be detected reasonably well -- but internally,
the distinction is very poor and their ordering is inconsistent. It is unclear
whether this is purely an arifact of the skewed distribution of rainfall,
or if it has a real basis in Yolngu definitions of seasons.


The tendency of this approach to show seasons interrupting each other at odd
times is a strength to exactly the degree that these interuptions accurately
reflect Yolngu understanding of the seasons.  Assessing this aspect of the
results would be a high priority for subsequent rounds of fieldwork, which
were not possible within the resources available for this thesis.

In the case that interruptions were too common, two solutions are proposed.
If phenological data were available, season definitions accounting for
ecological events might constrain the periods in which seasons can be detected.
This would be a fuller reflection of Yolngu knowledge, but faces serious
practical challenges -- and if historically typical timings are used rather
than live data, the results will fail to represent the `strange changes'
that participants reported seeing in their environment.
%
The second  possibility is to apply a more complex algorithm to weather data,
which prefers to detect longer runs of days for each season.  The disdvantages
of this approach include increased complexity, and either heightened sensitivity
to initial conditions or a loss of locality -- in either case, the result for
a certain day could be affected by observations at a substantial temporal distance.


\subsection{Quantitative Analysis}
\label{subsec:disc-season-characterisation}

\begin{figure}[p]
    \centerline{
    \includegraphics[width=1.1\linewidth]{galiwinku/timing-circle.pdf}}
    \caption[Three ways to characterise timing of seasons, Galiwinku]{
        Three ways to characterise timing of seasons, Galiwinku.
        The two rings in the center show the dominant season for each day,
        defined as that which was most often observed on that day (inner)
        and that with the greatest mean normalised index (outer).
        The arcs show intervals in which a specified proportion of days
        of that season fell -- thin line 90\%, medium 75\%, thick 50\% --
        which is the preferred approach to characterising season timing.
        The colour scheme is given by \cref{fig:galiwinku-season-pie}.
        }
    \label{fig:galiwinku-seasons}
\end{figure}

Part of the motivation for this research was dissatisfaction with the way in
which onset and typical timing of Indigneous seasons is typically presented
(if at all) -- usually by listing the months in which each season occurs.
%
An early draft suggested calculating the onset date of each season in each
year, and summarising this data.  Unfortunately, such summaries are less
meaningful when seasons can interrupt one another in any order; simply finding
the first date or last on which a season occurred is far from a robust method.

How then, if at all, can the typical timing of Yolngu seasons be accurately
characterised?  The results in \cref{sec:applying-seasons-method} suggest two
basic possibilities: for each day of the year, take the mean over all years
and compare either the mean normalised indicies, or the observed occurance
of each season.
%
Comparing the indicies has a conceptual clarity to it; it suggests that the
typical timing of a season is that period where conditions tend to fit the
definition of that season better than any other.  It can be estimated simply
by reading off the topmost line in \cref{fig:season-daily-index} for any
particular day.  This aggregation gives a steady progression for most of the
year, and while it jitters a little in the Wet no two seasons ever co-occur.

Comparing observed occurance (see \cref{fig:season-daily-prob}) is somewhat
further from the raw data, but closer to people's experience.  If a similar
result to the index-based characterisation is desired -- non-overlapping
occurance, albeit with some jitter -- the same approach of taking the season
with the highest value for each day of the year can be applied.
%
Alternatively, for each season we can calculate the shortest interval which
contains some proportion of all days on which that season occured -- eg.
ninety percent for a robust and overlapping intervals, or fifty percent for
shorter intervale with gaps between them.  In my opinion, this last option
is the most robust way to characterise typical timing of Yolngu seasons while
remaining well-founded in Indigenous definitions of the seasons.
%
All three approaches are drawn for comparison in \cref{fig:galiwinku-seasons}.



\section{Reflection -- methodology and limitations}
\label{sec:disc-reflection}

While the detail of specific methods is assessed above, reflections on the
methodology and scope are also important in exploratory research.  The
multi-stage approach was highly flexible, and this coupling between stages
made the study responsive to the unexpected qualitative results.  I hypothesise
that participant-led conversations and this flexibility were the primary
factors that led to this study's emergent focus on the complex structure of
Yolngu seasons -- an aspect of Indigenous seasonal knowledge which is neglected
by much of the literature.

I believe that the aim of this research -- to develop an approach to
integrating traditional and quantitative/analytical knowledge -- was well
served by the chosen methodology and methods.  Multi-stage and iterative
methods are well suited to exploratory research, and could form a strong
basis for further research.  Suggested improvements to the methods within
each stage, for investigation of Indigenous seasons specifically, are given
in the section above.  The scope of this thesis transitioned from grand
ambition to focussed inquiry as the depth and complexity of Yolngu seasonal
knowledge became clear.  The final scoping focusses on well-grounded analysis
of the data available, and leaves many tangents simply as suggested directions
for future research.

The limitations of this research are threefold:  the results barely scratch
the depth of Indigenous Knowledge, it is difficult to assess the robustness
of the findings, and only limited generalisation from them is possible.  These
limitations do not condemn the study, but urge cautious interpretation.
%
That there is much more to learn is not exactly a weakness of the research --
only so much ground can be covered in a few weeks of fieldwork -- but
encourages further study and careful thought.  With a small sample size and
the other limits described above, it is entirely possible that critical parts
of Yolngu seasonal knowledge are simple missing from the results above.
%
Difficulty in assessing the results is more concerning, and derived from
uncertainty as to what the results \textit{should} be.   Properly resourcing
follow-up fieldwork must be a priority for any future study; a second round
of interviews would allow correction of any issues in the qualitative results.
It could also validate the thresholds chosen in the numerical definitions of
seasons, or offer feedback as they were changed.  Most importantly it would
ensure that the research is a partnership between equals without requiring an
additional journey some time later.

The same detail and locality that embeds ecological and climate knowledge in
Indigenous calendars makes generalisation highly challenging.  Specific
findings of this study may not even be true for Yolngu calendars generally,
due to the focus on Galiwinku and Milingimbi.  More general findings, such as
the three-level typology of `kinds of seasons' may be more robust; for example
that pattern is recognisable in the Tiwi Seasons (\cref{fig:tiwi-seasons})
but would be surprising in temperate calendars.  Many of the premises, notably
including links between seasonal and ecological knowledge, are drawn from the
literature and appear to be reliable across a range of Indigenous groups.



\section{Directions for Further Study}
\label{sec:further-study}

Having demonstrated the viability of quantitative analysis of Indigenous
seasons, and discusses the value of such research, there are three distinct
directions for further study:  increasing one or more of the breadth, depth,
or extent of the research.  Alternative topics may also address the aim
of this research -- \textit{``developing an approach to integrating traditional
knowledge with a quantitative analytical approach''} -- but are not suggested
here.

\subsection{Breadth of Study}
Increasing the breadth of study would result in similar but less exploratory
research; essentially trading novelty for rigor.  Generally this means
collecting and analysing more data.

\begin{description}
\item[Multiple Indigenous calendars]
    Replication of this study with other Indigenous calendars would be
    valuable in each case, and the ability to compare calendars and investigate
    possible correlations between local climate and how seasons are defined
    could yield fascinating insights.

\item[Single calendar, multiple sites]
    Similarly, comparing the calendar of a single language group across
    multiple sites and subtly different climates could be very interesting.
    An `objective' weather record could really help researchers to draw out
    the differences between calendars, and perhaps see some of the subtleties.
    While this study had participants and weather observations from multiple
    sites, it did not analyse them as qualitatively distinct records or
    attempt any comparisons.

\item[Iterative fieldwork]
    Future work should if at all possible invest the time and resources
    neccessary for multiple visits to Indigenous homelands.  Second opinions,
    feedback on quantitative results, and the strong relationships that
    long-term engagement can grow could all prove surprisingly valuable.
\end{description}

\subsection{Depth of Study}
Deeper studies would likely focus on Indigenous Ecological Knowledge as it
relates to seasonal patterns, and work to integrate IEK with western scientific
knowledge.  Researchers must ensure that both Indigenous and non-Indigenous
people benefit from such work, as there is a considerable history of
exploitative `sharing' \citep[eg.][]{smith1999}.  Further analysis of monsoon-
or weather-based seasons would also fit in this category.

\begin{description}
\item[Kinds of seasons]
    \Cref{sec:complex-seasons} describes a typology of seasons -- monsoonal,
    meteorological, and ecological -- that can be identified as part of a
    `single' Yolngu calendar (to the degree that a `single yolngu calendar'
    is coherent).  Further study on the conceptual basis of this typology would
    be very interesting, as would gaining a stronger understanding of how
    seasonal indicators interact.

\item[Ecological knowledge]
    Investigating Indigenous ecological seasons could provide deep insights
    into local ecology and traditional natural resource management. The
    principal challenges are logistical:  such study would require a real
    investment of time and relationship-building, and the collecting data
    required to `translate' Indigenous Knowledge to western science would
    be expensive.

\item[Per-year patterns]
    There is clearly a degree of variation in the ways seasons occur between
    different years, study of which may reveal patterns.  Are there distinct
    regimes that describe seasonal onset, timing, or interruption?  If so, what
    data is required to categorise a year by regime and make useful predictions?
\end{description}

\subsection{Extent of Study}
Increasing the extend of study would mean expanding the topic; perhaps by
looking at links to the wider context -- climate indicies or long-term
change, non-Indigenous calendars, and so on.

\begin{description}
\item[Non-Indigenous calendars]
    A comparative study of many kinds of calendars could be fascinating --
    Indigenous, agricultural, solar, lunar, or harvest seasons -- unlocking
    the knowledge embedded in each.

\item[Climatology]
    With seasons defined by weather observations, a climatological study of the
    Yolngu calendar could suggest links between forecastable climate indices
    and Arnhem Land ecology.  Research could begin by investigating correlations
    between seasonal occurance and non-surface weather, or look at cycles with
    timescales on the order of months (eg. the Madden-Julian Oscillation), ENSO,
    the Indian Ocean Dipole, or even the Pacific Decadal Oscillation.

\item[Climate Change]
    The methods used in this study can be applied to any dataset which
    includes a daily record of the relevant variables.  This enables
    investigation of changes over time -- past or future -- and scenarios
    including a wider range of conditions than exist in the historical record.
    The obvious application of this is to investigate how Yolngu seasons have
    changed over the past century, and how they may change under future climate
    change scenarios in model forecast and backcasts.  There is substantial
    interest in such a study from Yolngu participants.
\end{description}



\section{Applications and Benefits of the Research}
\label{sec:applications-benefits}

\todo{remove subsection headings; just to help drafting}


\subsection{Concrete Applications}

Seasons provide powerful heuristics for the expected range and variability in
weather for a given time of year.

Implications of seasonality, relationship to NRM, applications, meaning of
change, what does this way of thinking translate into action
relevance to people's lives, industry, etc.

Quantitative knowledge of local indigenous seasonality can improve practises in
natural resource management, fire control, agriculture, and tourism -
particularly outside the temperate zone.  Seasons can provide powerful
heuristics for the expected range and variability in weather, and may yield
insights into patterns of change.  For practical use, a local and long-tested
seasonal calendar has substantial advantages over one imported from Europe.

For the Yolngu community, increased engagement with the science around climate
impacts has benefits for both adaptation choices and advocacy for mitigation,
which are particularly important for coastal communities living off the land.
Yolngu have expressed strong interest in other records of and perspectives on
past environmental changes.

what activities in northern australia
in particular (eg land and NR management) will benefit from these insights?
Key aspect: how transferable are these insights?



\subsection{Academic Contribution}

This research is expected to make a novel contribution to knowledge, raise
awareness of Yolngu culture, and improve understanding of the climate
seasonality of NE Arnhem Land.  The methods demonstrate two-ways research in
combining indigenous knowledge with numerical data to characterise seasons.



\subsection{Intangible Benefits / Facing the Anthropocene}

\todo{Thinking about `this research can inspire action' to cross knowledge boundaries
-- widens scope back out to mirror intro!}


For non-technical communities, both Indigenous and non, discussing climate
change in terms of seasonality is more meaningful than degrees of warming.
Framing the impacts of climate change in terms relatable to lived experience
has the potential to boost engagement with and understanding of the major
results of climate science - after all, everyone has some experience of the
weather!

For non-indigenous Australians, this research contributes to the continuing
process of learning to live on this continent, discarding `maladaptive traditions'
- eg our poor infrastructue planning for drought and fire (REF the future eaters).



