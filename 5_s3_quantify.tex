\chapter{Section 3:  Creating a Quantified Yolngu Calendar}
The mixed methods, results, and interpretive discussion.

\section{Methods}
-	Briefly describe interview technique to elicit details, how coded in transcripts/notes
-	Touch on literature for seasonal onset detection (eg monsoon).  Stats with time series and missing data are also relevant.
-	Three approaches minimum:  traditional, and others discussed with SCU
-	Describe statistical approaches and meta-method for using them all
-	Outline how to select, normalise, check input data (eg comparison across stations)

\section{Results and Discussion}
1.	The six seasons are <> (see above).  They are defined by <>.  This calendar has the following interesting properties:  (eg – variable onset, multiple occurrence, at-least-once by climate not construction).
2.	Summary statistics (as charts) for input data, ie weather observations.
3.	For each season:
a.	Determine seasons by each approach, and discuss reliability
4.	<approach> is best, for <reasons>.  Finer distinctions if applicable.
5.	The Quantified calendar is as follows:  <> (with detection functions)


