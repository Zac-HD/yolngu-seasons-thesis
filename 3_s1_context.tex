\chapter{Section 1: Context and importance of this research}
This section does a number of things, all of them focussed on setting up later sections.

\section{Context}
Why does this research matter?  Why do it now?

\section{Literature Review}
After -detailed search strategy-, I was unable to identify any studies which quantified indigenous Australian seasons.
I thus review a number of related fields, explaining the relevance of each and drawing together a summary of this 'gap at the intersection'.
MUST have table here pointing to other parts of the literature review, explaining that concrete parts are in the related methods section to avoid misunderstanding. 
Possible sections – to be reorganised and mutated:

\begin{itemize}
\item Cross cultural and qualitative methods
\item Tropical seasonality, including monsoon and ENSO
\item Anticipated benefits and applications of the research
\item Relationship to phenology and historical climate change
\item Indigenous seasons literature (summary, also compare and contrast)
\item Overview of calendars and seasons – what are they and how are they defined?
\end{itemize}

\section{Methodology and Research Framework}
Explain the structure and relationship of sections 2-4, and how they tie together.  Methods are in each section, but a coherent methodology for the whole thing goes here.  Includes the reason for sections and for each section, ie overall strategy that the research executes.
